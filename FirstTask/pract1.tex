\section*{\LARGE{Цель практической работы}}
В этой практической работе рассмотрим создание первого приложения для Android.
Научимся создавать проекты и запускать приложение в режиме отладки.
Так же узнаем об основах разработки приложения для Android,
включая создание простого пользовательского интерфейса и обработку ввода информации пользователем.

\newpage

\section*{\LARGE{Выполнение практической работы}}

\section{Создание проекта в Android Studio}
Чтобы создать новый проект в Android Studio, надо отрыть данное приложение.
Щелкнуть на кнопку New Project на экране приветствия. Или если открыт другой проект, щелкнуть меню File и выберать New Project.
В передложенном списке стартовых activity выбираем наиболее подходящий и далее необходимо заполнить поля в окне Configure your new project.

\section{Создание разметки}
Создание разметки в XML файлах предпочтительнее, чем в исходном коде по нескольким причинам, 
но главным образом из-за необходимости создания различных файлов разметки для устройств с различными размерами экрана.

\begin{figure}[hp]
  \begin{center}
    %\includegraphics{Screenshot from 2023-02-12 15-28-03.png}
    \caption{очень важная картинка}
    \label{fig:}
  \end{center}
\end{figure}

\subsection{Создание линейной разметки (Linear Layout)}
\begin{enumerate}
	\item В Android Studio открыли файл res/layout/activity\_my.xml
	\item В окне предпросмотра щелкнули по иконке Hide , чтобы скрыть окно.
	\item Удалили элемент <TextView>.
	\item Изменили элемент <RelativeLayout> на <LinearLayout>.
	\item Добавили атрибут android:orientation и установите для него значение "horizontal".
	\item Удалите атрибуты android:padding и tools:context.
\end{enumerate}

\subsection{Добавление текстового поля}
Как и для каждого объекта типа View, необходимо указать некоторые XML атрибуты, характерные для элемента EditText.

\newpage

\subsection*{Вывод}
В ходе практической работы было создано первое приложение для Android. Научились создавать проекты и запускать приложение в режиме отладки. Так же узнали об основах разработки приложения для Android, такие как создание разметки с текстовыми полями, полями ввода текста и кнопки, и обработку ввода информации пользователем.
