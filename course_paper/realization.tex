\chapter{РАЗДЕЛ РЕАЛИЗАЦИЯ ПРОЕКТА}
\section{Дизайн мобильного приложения}
Итеративный процесс улучшения дизайна.
Создание дизайна --- это итеративный процесс,
который может потребовать нескольких попыток,
чтобы достичь наилучших результатов.
Различные элементы могут быть изменены и изменены на основе обратной связи
пользователей, чтобы улучшить пользовательский интерфейс
и создать максимально удобный и привлекательный дизайн приложения.\par
В данной курсовой работы в том числе необходимо продумать дизайн
мобильного приложения. По заданной теме разрабатывалось
приложение для чтения электронных книг.\par
Цветовая гамма выбрана в серых тонах, чтобы создать так называемую "<темную
тему">. Что снизит нагрузку на глаза пользователя, а значит повысит удобство
приложения.\par
В нижней части экрана телефона или приложения располагается
панель навигации с четырьмя кнопками: настройки, текущая книга,
библиотека, заметки. Соответственно, эти четыре окна можно считать главными
для приложения, относительно остальных. А значит остальные окна будет
дочерними к ним.\par
Создание дизайна приложения в Android Studio включает в себя несколько шагов:

\begin{itemize}
	\item Создание макета (Layout) для экрана приложения.
		Это можно сделать через интерфейсный конструктор в Android Studio,
		добавляя различные элементы интерфейса, такие как кнопки,
		текстовые поля и изображения, на поверхность макета.
	\item Использование стилей (Style) для элементов интерфейса.
		Это позволяет задать цвета, шрифты, размеры и дизайн
		для элементов интерфейса, чтобы они выглядели консистентно
		и соответствовали общему стилю приложения.
	\item Использование тем (Theme) для макетов и стилей.
		Темы позволяют задать общие свойства для всего приложения,
		такие как цвет заднего фона, шрифты и стили.
	\item Работа с ресурсами (Resource) приложения.
		Ресурсы, такие как изображения, цвета и макеты,
		могут быть добавлены в проект и использованы в дизайне приложения.
\end{itemize}
\subsection{Окно "<Настройки">}
В данном окне пользователю предоставляется возможность настроить
приложение под него (рис.~\ref{fig:w:settings}).
Здесь размещается ползунок для настройки яркости экрана,
которая была разработана в курсовой работе, предыдущего семестра.

\begin{image}
	\includegrph{Screenshot from 2023-05-26 13-43-56}
	\caption{Пример окна "<Окно чтения">}
	\label{fig:w:settings}
\end{image}

\subsection{Окно "<Библиотека">}
В этом окне, располагается библиотека, представляющая собой
список книг в два вертикальных ряда.
Элемент списка или книга содержит обложку, это первая страница книги, и
ee название (рис.~\ref{fig:w:library}).

\begin{image}
	\includegrph{Screenshot from 2023-05-26 13-44-03}
	\caption{Пример окна "<Библиотека">}
	\label{fig:w:library}
\end{image}

При нажатии на книгу из списка пользователя перенаправляет а окно
чтения.\par

\subsection{Окно "<Текущая книга">}
В этом окне отображается:
\begin{itemize}
	\item книга, которую пользователь читает в данный момент;
	\item заметки о книге, в которые пользователь сможет записать
		свое мнение о книге, важные цитаты или то что он захочет;
	\item полоса прогресса чтения книги.
\end{itemize}

Окно иллюстрирует рисунок~\ref{fig:w:cur-book}.\par
Информация о книга, которую пользователь читал
последней, хранится в базе данных и, аналогично,
хранятся заметки, связанные с книгой.

\begin{image}
	\includegrph{Screenshot from 2023-05-26 13-44-00}
	\caption{Пример окна "<Текущая книга">}
	\label{fig:w:cur-book}
\end{image}

Как и в окне "<библиотека">, при нажатии на книгу, пользователя
перенаправляет в окно чтения. Аналогичным образом,
при нажатии на заметку, пользователя перенаправлят на страницу
редоктирования заметки.

\subsubsection{Окно чтения}
Данное окно предназначено для чтения книги (рис.~\ref{fig:w:reading}).
В нем можно пролистывать, уменьшать и увеличивать маштаб страницы.
В это окно можно прейти двумя способами. Нажать на книгу в
окне "<библиотеки"> или в окне "<текущая книга">.

\begin{image}
	\includegrph{Screenshot from 2023-05-26 21-43-24}
	\caption{Пример окна для чтения}
	\label{fig:w:reading}
\end{image}

Здесь, сверху расположены две кнопки, левая возвращает в окно
"<текущая книга">, а правая открывает меню в котором есть:

\begin{itemize}
	\item "<Инфо">, выводит окно с информацией (рис~\ref{fig:w:book:info});
	\item "<Дабавить заметку">, переносит пользователя в окно добавления
		заметки;
	\item "<Закладки">, переводит пользователя в окно с закладками книги
		(глава~\ref{sct:bookmark}).
\end{itemize}

\begin{image}
	\includegrph{Screenshot from 2023-05-26 21-43-28}
	\caption{Пример вывода информации о книге}
	\label{fig:w:book:info}
\end{image}

\subsubsection{Окно "<Закладки">}\label{sct:bookmark}
В данном окне предостовляется список закладок к книге
(рис.~\ref{fig:w:book:bookmark}).\par
В верхней части находятся поля ввода для текста закладки и номера страницы.
Ниже идут кнопки "<добавить"> и "<удалить">. После нажатия на кнопку
"<удалить">, нажимая на заметки они удаляются,
чтобы выйти из этого режима, нужно нажать на кнопку "<готово">,
находящююся на прежнем месте кнопки "<удалить">.

\begin{image}
	\includegrph{Screenshot from 2023-05-26 21-44-13}
	\caption{Пример окна "<Закладки">}
	\label{fig:w:book:bookmark}
\end{image}

\subsection{Окно "<Заметки">}
В этом окне, предоставляется список всех заметок, которые были
созданы пользователем (рис.~\ref{fig:w:notes}).
Так же, как и в окне с текущей книгой, в будущем, предполагается
хранить заметки в базе данных или в виде файлов в каталоге приложения.

\begin{image}
	\includegrph{Screenshot from 2023-05-26 13-44-06}
	\caption{Пример окна "<Заметки">}
	\label{fig:w:notes}
\end{image}

В правом нижнем углу окна располагается кнопка добавления записи.
Она перенаправляет пользователя в окно, в котором создается новая заметка
и добавляется в список (глава~\ref{sct:note:add}).
При нажатии на любую из записей пользователя
перенаправляет в окно редактирования заметки (глава~\ref{sct:note:change}).

\subsubsection{Окно добавления заметки}\label{sct:note:add}
Пользователю в этом окне предлагается ввести заголовок для заметки и
ее текст. Внизу, на месте, где была навигационная панель, находиться
кнопка сохранения этой заметки (рис.~\ref{fig:w:notes:add}).\par

\begin{image}
	\includegrph{Screenshot from 2023-05-26 13-43-47}
	\caption{Пример окна "<Окно чтения">}
	\label{fig:w:notes:add}
\end{image}

После нажатия на кнопку "<сохранить">, пользователя перенаправляет в
окно со списком заметок, в который добавляется только что сохраненная.

\subsubsection{Окно редактирования заметки}\label{sct:note:change}
Это окно выглядит практически также, как окно добавления заметки, за
исключением еще одной кнопки для удаления записи
(рис.~\ref{fig:w:notes:change}).\par

\begin{image}
	\includegrph{Screenshot from 2023-05-26 13-44-10}
	\caption{Пример окна "<Окно чтения">}
	\label{fig:w:notes:change}
\end{image}

При попадании в это окно поля заголовка и текста заполнены, в них
находиться содержимое заметки, которое пользователь вводил при ее
создании.\par
Кнопка удаления заметку, удаляет заметку из списка, а кнопка
сохранить, соответственно сохраняет в списку. И обе кнопки перенаправляют
пользователя в окно со списком записей.

\section{База данных}
SQLite является встраиваемой базой данных на устройствах Android.
Она используется для хранения и управления информацией,
связанной с приложениями на Android-устройствах. \par
SQLite обладает преимуществами, такими как производительность и малый размер,
что делает его идеальным выбором для мобильных приложений.
С помощью SQLite можно эффективно организовать, хранить и управлять данными
в приложении --- это очень важно, особенно если приложение
содержит множество данных и пользовательских настроек.\par
В данной курсовой работе, связанной с разработкой приложения под Android,
использование SQLite полезно для создания базы данных,
которая будет хранить информацию о книгах, закладках и заметках.\par

Таким образом, использование SQLite в курсовой работе обеспечит
эффективное управление данными в приложении и позволит создать
базу данных с удобным и понятным интерфейсом для работы с ней.

\subsection{DBHelper}
Для начала создали класс DBHelper, наследуемый от класса SQLiteOpenHelper,
который предоставляет методы для создания, обновления и удаления
таблиц в базе данных.\par
DBHelper --- это класс-помощник по управлению базой данных SQLite.
Он предоставляет методы для создания, обновления и управления базой данных.
Этот класс упрощает работу с базой данных и предоставляет функциональность
для выполнения запросов к базе данных.\par
Далее переопределили метод onCreate().
Добавить код для подключения к базе данных в методе onCreate() класса,
который наследует интерфейс SQLiteOpenHelper (листнинг~\ref{lst:db:helper}).

\subsection{Модели}
Модели при создании взаимодействия с базой данных sqlite
в приложении для Android используются для представления данных,
которые хранятся в базе данных. Они позволяют легко
работать с данными в приложении, а также обеспечивают простой
и интуитивно понятный способ доступа к данным. \par
Модели представляют собой классы данных,
которые содержат свойства для каждого поля в таблице базы данных.\par
Использование моделей в приложении для Android предоставляет
следующие преимущества:

\begin{itemize}
	\item Облегчение работы с данными в приложении.
		Модели упрощают структуру данных и позволяют более
		эффективно оперировать ими в коде приложения.
	\item Увеличение безопасности при работе с данными.
		Модели позволяют полностью контролировать типы данных
		и обрабатывать ошибки в случае неправильного ввода или вывода данных.
	\item Улучшение производительности приложения.
		Модели позволяют выгружать только нужные данные из базы данных,
		что повышает скорость работы приложения
		и уменьшает нагрузку на систему.
	\item Упрощение разработки приложения.
		Использование моделей позволяет разделить код приложения
		на более логические блоки, что упрощает разработку,
		поддержку и расширение приложения.
\end{itemize}

В данной курсовой работе использовалось три модели: книга, заметка и закладка
(листнинги~\ref{lst:db:model:book}\,---\,\ref{lst:db:model:bookmark}).

\subsection{Адаптеры}
Адаптер для DBHelper в Android используется для связи между данными
и пользовательским интерфейсом, который может отображать эти данные.
Он может использоваться для заполнения списка, таблицы или
другого представления элементов, которые были извлечены из базы данных.
Адаптеры помогают стандартизировать взаимодействие между представлением
и данными, упрощают программирование
и повышают производительность приложения.

\subsubsection{Создание BookDBAdapter}
BookDBAdapter необходим для управления таблицы, в которой хранится информация
о книгах, базы данных SQLite (листнинг~\ref{lst:db:adapter:book}).
Этот класс позволяет приложению выполнять операции
с базой данных, такие как:

\begin{itemize}
	\item добавление книги;
	\item обновление книги;
	\item удаление книги; 
	\item получение всех книг;
	\item получение конкретной книги по ее индетификатору;
	\item получить книгу, которую читали дольше всех;
	\item проверка пуста ли таблица;
	\item получить словарь: название-индетификатор.
\end{itemize}

BookDBAdapter также обеспечивает конкретную реализацию методов
для работы с базой данных, таких как открытие и закрытие базы данных,
выборка данных из таблицы книг.\par
Именно через BookDBAdapter приложение получает информацию о книгах,
которые уже были добавлены в базу данных, а также добавлять новые книги,
что значительно упрощает работу с базой данных.
Без BookDBAdapter было бы гораздо сложнее хранить
и управлять большим количеством информации о книгах в приложении.

\subsubsection{Создание NoteDBAdapter}
NoteDBAdapter в данной курсовой работе нужен для управления таблицей
с заметками в базе данных SQLite
(листнинг~\ref{lst:db:adapter:note}).\par
Это позволит сохранять и загружать заметки для каждой книги,
которую пользователь открывает в приложении.
Кроме того, NoteDBAdapter позволит легко выполнить операции
CRUD («создать», «читать», «обновить» и «удалить») на записи заметок
в базе данных, а также поиск и сортировку заметок для определенной книги.

\subsubsection{Создание BookMarksDBAdapter}
BookMarksDBAdapter нужен для управления таблицей в базе данных SQLite,
которая хранит закладки, добавленные пользователем в приложении
(листнинг~\ref{lst:db:adapter:bookmark}).\par
Он облегчает доступ к базе данных и позволяет выполнять
операции по:

\begin{itemize}
	\item добавлению закладки;
	\item чтению закладки;
	\item получение всех по индетификатору книги. 
\end{itemize}

С помощью BookMarksDBAdapter можно без проблем управлять таблицами
базы данных и выполнить множество операций со связанными данными
в приложении для чтения книг, обеспечивая лучшую работоспособность
и надежность приложения.

