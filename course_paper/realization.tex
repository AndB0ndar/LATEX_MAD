\chapter{РАЗДЕЛ РЕАЛИЗАЦИЯ ПРОЕКТА}

\section{Дизайн мобильного приложения}
В данной курсовой работы необходимо было продумать дизайн
будущего мобильного приложения. По заданной теме разрабатывалось
приложение для чтения электронных книг.\par
Цветовая гамма выбрана в серых тонах, чтобы имитировать «темную
тему». Что снизит нагрузку на глаза пользователя, а значит повысит удобство
приложения.\par
В нижней части экрана телефона или приложения будет располагаться
панель навигации с четырьмя кнопками: настройки, текущая книга,
библиотека, заметки. Соответственно, эти четыре окна являются главными
для приложения, относительно остальных. Остальные окна будет дочерними
к ним. Теперь разберем каждое главное окно вместе с дочерними к нему
окнами.

\subsection{Окно «библиотека»}
На первом окне, который пользователь будет видеть при открытии
приложения, будет располагаться библиотека. Библиотек представляет собой
вертикальный список книг. Элемент списка или книга содержит обложку и
минимальную информацию о ней, название книги и автор этой книги
(Рисунок~\ref{fig:w:library}).

\begin{image}
	\includegrph{img-023}
	\caption{Пример окна "<Библиотека">}
	\label{fig:w:library}
\end{image}

При нажатии на книгу из списка пользователя перенаправляет а окно
чтения.\par

\section{Окно «Текущая книга»}
В этом окне отображается книга, которую пользователь читает в
данный момент, и заметки о книге, в которые пользователь сможет записать
свое мнение о книге, важные цитаты или то что он захочет
(Рисунок~\ref{fig:w:cur-book}).\par
В будущем, информация о книга, которую пользователь читал
последней, будет храниться в базе данных или файлах приложения,
аналогично будут хранится заметки, связанные с книгой.

\begin{image}
	\includegrph{img-025}
	\caption{Пример окна "<Текущая книга">}
	\label{fig:w:cur-book}
\end{image}

Как и в окне «библиотека», при нажатии на книгу, пользователя
перенаправляет в окно чтения.

\section{}

