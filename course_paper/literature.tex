\begin{thebibliography}{00}
	\bibitem{} Авдеев В.А. Организация ЭВМ и перифирия с демонстрацией
		имитационных моделей.~--- М.: ДМК, 2014.~--- 708~с.
	\bibitem{} Антамошкин, О. А. Программная инженерия. Теория и практика.
		Учебник. M: НИЦ Инфра-М, 2012.~--- 368~с.
	\bibitem{} Дейтел, Х., М. Операционные системы.
		Основы и принципы. Т. 1~--- М.: Бином, 2016.~--- 1024~c.
	\bibitem{} Дейтел, Х., М. Операционные системы. Т. 2.
		Распределенные системы, сети, безопасность. М.: Бином, 2016.~--- 704~c.
	\bibitem{} Таненбаум, Э. Современные операционные системы
		/ Э. Таненбаум. СПб.: Питер, 2013.~--- 1120~c.
	\bibitem{} С.А.Орлов. Программная инженерия.
		Учебник для вузов. 5-е издание обновленное и дополненное.М:
		Издательский дом "<Питер">, 2017. 812~с.
	\bibitem{} Дейв Тейлор. Сценарии командной оболочки.
		Linux, OS X и Unix. 2-е издание.
		Издательский дом "<Питер">, 2017. 624~с.
	\bibitem{} SWEBOK V2, 2004~г.
	\bibitem{} SWEBOK V3, 2013~г.
\end{thebibliography}

