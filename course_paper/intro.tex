\chapter*{ВВЕДЕНИЕ}
\addcontentsline{toc}{chapter}{ВВЕДЕНИЕ}

В наше времена чтение уходят на второй план и очень зря.\par
Благодаря книгам человек саморазвивается, обогащается его словарный
запас, речь совершенствуется. Читая литературное произведение, людям
автоматически приходится запоминать имена героев, события в
хронологической последовательности, а также предсказывать развитие
сюжетной линии.\par
Это происходит из-за того, что читать бумажную книгу бывает не очень
удобно, например, ее просто негде нести, чтобы почитать по дороге на работу
или в институт. Проблему может решить электронная книга, она портативнее
и может содержать в себе сотни книг. Но и в ней есть важные минусы, во-
первых это еще одно электронное устройство, которое придется заряжать,
вместе с телефоном, планшетом и ноутбуком. Во-вторых, не каждый готов
покупать устройство для чтения книг, которое в несколько десятков раз
дороже самих книг.\par
Идеальный вариант использовать свой мобильный телефон, как
электронную книгу. Именно для этого и будет предназначен программный
комплекс --- домашняя библиотека.\par
В данной курсовой работе рассматривается создание программного комплекса
для чтения книг на языке программирования Java.\par
Будет проведен анализ существующих программ для чтения,
определены основные требования к разрабатываемому комплексу,
а также описан метод разработки программы.\par
Разработка программного комплекса на Java должна упростить
процесс чтения книг и предоставить пользователям удобство и
комфорт в использовании.

