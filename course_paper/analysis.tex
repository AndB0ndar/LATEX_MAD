\graphicspath{{./img}}
\chapter{РАЗДЕЛ АНАЛИЗА И ИССЛЕДОВАНИЙ}
\section{Выбор языка разработки}
Java является одним из самых популярных языков программирования
для разработки мобильных приложений и имеет множество преимуществ
для создания приложений для чтения книг на Android.
Вот несколько причин, почему Java хороший выбор для разработки
мобильного приложения для чтения книг на Android:

\begin{enumerate}
	\item Поддержка Android: Java является официальным языком
		программирования для разработки приложений на Android,
		поэтому он обладает лучшей поддержкой для этой платформы.
	\item Платформно-независимость: Java позволяет создавать приложения,
		которые могут работать на разных платформах и операционных системах,
		что делает его идеальным выбором для создания приложений для Android.
	\item Кросс-платформенность: Java является кросс-платформенным языком
		программирования, что означает, что можно использовать один
		и тот же код для создания приложений для разных устройств.
	\item Мощная библиотека: Java имеет обширную библиотеку
		для создания мобильных приложений, что позволяет
		разработчикам легко добавлять новые функции в приложение
		для чтения книг на Android.
	\item Распространенность: Java очень распространен в мире
		разработки программного обеспечения, что означает,
		что вы можете легко найти поддержку и решать возникающие проблемы,
		а также обмениваться знаниями с другими разработчиками.
	\item Простота: Java является одним из наиболее простых языков
		программирования, доступных для начинающих разработчиков,
		и имеет множество инструментов и ресурсов для обучения.
\end{enumerate}

В целом, Java является одним из лучших языков программирования
для разработки мобильных приложений под Android,
благодаря своей поддержке, кроссплатформенности,
производительности и безопасности.

\section{Дизайн приложения}
Мобильные приложения прочно вошли в жизнь людей. Популярность
мобильных телефонов привела к тому, что мобильные приложения стали
важнейшей частью смартфона, может даже важнее возможности звонить по
нему. Ими уже сложно удивить, так что, чтобы завоевать популярность среди
прочих приложений, дизайн должен быть проработан до мелочей.\par
Программа должна быть визуально приятной и простой в
использовании.\par
В самом начале, необходимо провести анализ конкурентов и задач,
которые должна решать программа. Чтобы дизайн был удобен, главное что
нужно это продумать перемещение пользователя между окнами приложения.
Затем, при проектировании интерфейса, необходимо учитывать несколько
деталей:

\begin{enumerate}
	\item Элементы управления, которые используются чаще остальных,
		расположить внизу экрана. В противоположность, в верхнем левом углу,
		кнопки должны использоваться реже остальных.
	\item Должна быть возможность вертикально скролить приложение,
		чтобы пользователь мог расположить информацию на удобном ему уровне.
		В свою очередь, для разработчика это позволит увеличить объем
		предоставляемой информации на странице.
	\item Не стоит дублировать логотип приложения на всех экранах. Будет
		достаточно, отобразить на основной иконке и экране загрузки.
	\item Шрифт должен быть легко читаемым.
	\item Цвета должны быть спокойными, не режущими глаз. А также они
		должны быть нативными, то есть подсказывать пользователю, на какие
		элементы можно кликать, а на какие нет. Также цветом можно
		разграничить разные по смыслу элементы.
\end{enumerate}

\section{Анализ предметной области}
Объектом разработки данной курсовой работы является приложение,
позволяющее сохранять файлы книг на мобильном устройстве, сортировать в
библиотеке и читать их.\par
Все книги разделены на категории, такие как: прочитанные или
непрочитанные, фантастка или документалистика. Это позволит пользователю
проще искать подходящую для него книгу в библиотеке. При чтении книги
бывает полезно сохранить важные цитаты или свои мысли о прочитанном.
Для этого в приложении предусмотренно создание заметок, которые могут
как быть связаны или нет с конкретной книгой.\par
Конечно, в современных маркетплейсах можно найти много
приложений-читалок. Рассмотрим некоторые бесплатные продукты-аналоги
с Google Play.\par
Мобильное приложение eBox~--- простая читалка книг: fb2, pdf, docx, txt.
Он содержит такой функционал как: отслеживание прогресса чтения,
настройки шрифтов и яркости экрана, оффлайн чтение, возможность перехода
по ссылкам и сноскам.\par
Экранные формы приложения приведены на рисунке~\ref{fig:app:ebox}.

\begin{image}
	\includegrph[width=0.6\textwidth]{oapps-ebox}
	\caption{Экраны мобильного приложения eBox}
	\label{fig:app:ebox}
\end{image}

Мобильное приложение KyBook~2~--- простая читалка книг: fb2, pdf, mp4,
modi, txt. Его ключевые возможности: OPDS каталоги, облачное хранилище,
гибкие настройки режима чтения, удобный каталогизатор. Экранные формы
приложения приведены на рисунке~\ref{fig:app:kybook}.

\begin{image}
	\includegrph[width=0.6\textwidth]{oapps-kybook}
	\caption{Экраны мобильного приложения KyBook 2}
	\label{fig:app:kybook}
\end{image}

Некоторые другие приложения из Google Play обладают недостатки
такими как: неудобная навигация, невозможность настройки, необходимость
Интернет-соединения, отсутствие русского языка. Интуитивность навигации
и возможность настройки приложения увеличивают качество приложения и
приятность работы с ним. Возможность работы в оффлайн режиме также
немаловажна из-за того, что мобильный Интернет роуминге сильно дорогой.
Стоить отметить, что рассматривались только бесплатные приложения,
которые более доступны для пользователей. Приложение, разрабатываемое в
ходе дипломной работы, призвано решить проблемы, возникающие при
работе с продуктами аналогами.\par
К разработке выдвигаются следующие требования:

\begin{itemize}
	\item необходимо реализовать поиск книг по категориям;
	\item должна быть возможность создания заметок, некоторые из
		могут быть привязаны к конкретным книгам. Которые должны
		храниться в базе данных;
	\item пользователь должен иметь возможность настраивать приложение
		под свои потребности;
	\item приложение должно функционировать без наличия Интернет-соединения.
\end{itemize}

Таким образом, в данной статье рассмотрена актуальность разработки
приложения-читалки книг для мобильного устройства, в частности устройства
под управлением операционной системы Android.

\section{Постановка задачи}
Данная выпускная курсовая работа посвящена разработке мобильного
приложения для чтения электронных книг. Приложение создается для
автоматизации процесса хранения электронных книг и их чтения с помощью
смартфона на платформе Android.\par
Чтобы программа была ориентирована на пользователя, она должна
обладать удобным и интуитивно понятном интерфейсом, обеспечивающий
простой и комфортный доступ к функциям приложения. Для надежного
хранения и обработки информации необходимо разработать базу данных (БД),
содержащую книги, информацию о книгах и заметки пользователя.\par
Мобильное приложение должно обладать следующим функционалом:

\begin{itemize}
	\item настройка приложения;
	\item организация списка книг;
	\item просмотр содержимого файла книги (чтение книги);
	\item добавление и удаление книг в приложение;
	\item создание, изменение и удаление заметок.
\end{itemize}

