\chapter*{ЗАКЛЮЧЕНИЕ}
\addcontentsline{toc}{chapter}{ЗАКЛЮЧЕНИЕ}
В ходе выполнения курсовой работы было реализовано
мобильное приложение "<домашняя библиотека"> на языке программирования Java
для операционной системы Android.\par
Было проведено исследование существующих решений и технологий, используемых
для разработки мобильных приложений на платформе Android.
Были рассмотрены различные аспекты проектирования и разработки приложений,
включая выбор архитектуры, используемые инструменты и фреймворки.\par
На основе проведенного анализа было выбрано использование
языка программирования Java и платформы Android для разработки
мобильного приложения для чтения книг. Были выбраны определенные инструменты,
такие как Android Studio и различные библиотеки
для реализации функционала приложения.\par
Был разработан дизайн для мобильного приложения. Дизайн
приложения предназначен для облегчения взаимодействия пользователя с ним
и определяет его узнаваемость среди прочих. Еще, важно чтобы он следовал
современным трендам в разработке.
В результате разработано мобильное приложение для чтения книг,
которое позволяет пользователю удобно читать электронные книги
на устройствах под управлением операционной системы Android.\par
Таким образом, выполнение данной курсовой работы позволило познакомиться
с основными концепциями и технологиями,
используемыми при разработке мобильных приложений на платформе Android,
а также практически применить полученные знания
при создании функционального приложения для чтения книг на Java.

