\graphicspath{{./seventh/img}}

\section*{\LARGE{Введение}}
\addcontentsline{toc}{section}{Введение}
\textbf{Цель}: научиться работать с файлом манифеста AndroidManifest.
А также розобраться с передачей данных между Activity и получение
результатов из Activity.\par

\clearpage

\section*{\LARGE{Выполнение практической работы}}
\addcontentsline{toc}{section}{Выполнение практической работы}

\section{Простой файл манифеста}
Элементом корневого уровня является узел manifest. В данном случае только
определяется пакет приложения --- \texttt{package="com.example.viewapp"}.
Собственно, это определение файла манифеста по умолчанию. В каждом
конкретном случае может отличаться пакет приложения, остальное
содержимое при создании проекта с пустой activity будет аналогичным.\par
Большинство настроек уровня приложения определяются элементом
application. Ряд настроек задаются с помощью атрибутов. По умолчанию
применяются следующие атрибуты:
\begin{itemize}
	\item \texttt{android:allowBackup} указывает, будет ли для приложения
		создаваться резервная копия. Значение
		\texttt{android:allowBackup="true"} разрешает создание
		резервной копии;
	\item \texttt{android:icon} устанавливает иконку приложения.
		При значении \texttt{android:icon="@mipmap/ic\_launcher"} иконка
		приложения берется из каталога res/mipmap;
	\item \texttt{android:roundIcon} устанавливает круглую иконку приложения.
		Также берется из каталога \textbf{res/mipmap};
	\item \texttt{android:label} задает название приложение, которое будет
		отображаться на мобильном устройстве в списке приложений и в
		заголовке. В данном случае оно хранится в строковых ресурсах -
		\texttt{android:label="@string/app\_name"};
	\item \texttt{android:supportsRtl} указывает, могут ли использоваться
		различные RTL API --- специальные API для работы с правосторонней
		ориентацией текста (например, для таких языков
		как арабский или фарси);
	\item \texttt{android:theme} устанавливает тему приложения.
		Тема определяет общий стиль приложения. Значение
		\texttt{@style/Theme.ViewApp"} берет тему \texttt{"Theme.ViewApp"} из
		каталога \textbf{res/values/themes}.
\end{itemize}

Пример использования показан в листнинге~\ref{lst:manifest:simple}.

\begin{lstlisting}[language=XML
	, caption=\leftline{}
	, label=lst:manifest:simple
	]
<application
	android:allowBackup="true"
	android:dataExtractionRules="@xml/data_extraction_rules"
	android:fullBackupContent="@xml/backup_rules"
	android:icon="@mipmap/ic_launcher"
	android:label="@string/app_name"
	android:roundIcon="@mipmap/ic_launcher_round"
	android:supportsRtl="true"
	android:theme="@style/Theme.Seventh"
	tools:targetApi="31">
	<activity
		android:name=".MainActivity"
		android:exported="true">
		<intent-filter>
			<action android:name="android.intent.action.MAIN" />

			<category android:name="android.intent.category.LAUNCHER" />
		</intent-filter>

		<meta-data
			android:name="android.app.lib_name"
			android:value="" />
	</activity>
</application>
\end{lstlisting}

Элемент \texttt{intent-filter} в MainActivity указывает,
как данная activity будет использоваться. В частности, с помощью узла
\texttt{action android:name="android.intent.action.MAIN"},
что данная activity будет
входной точкой в приложение и не должна получать какие-либо данные извне.\par
Элемент \texttt{category android:name="android.intent.category.LAUNCHER"}
указывает, что MainActivity будет представлять стартовый экран, который
отображается при запуске приложения.\par
Файл манифеста может содержать множество элементов, которые имеют
множество атрибутов. И все возможные элементы, и их атрибуты можно
найти в документации. Здесь же рассмотрим некоторые примеры
использования.

\section{Определение версии}
С помощью атрибутов элемента manifest можно определить версию
приложения и его кода.\par
Пример использования показан в листнинге~\ref{lst:version}.

\begin{lstlisting}[language=XML
	, caption=\leftline{}
	, label=lst:version
	]
<manifest
    xmlns:android="http://schemas.android.com/apk/res/android"
    xmlns:tools="http://schemas.android.com/tools"
    android:versionName="1.0"
    android:versionCode="1">

	...

</manifest>
\end{lstlisting}

Атрибут \texttt{android:versionName} указывает на номер версии, который будет
отображаться пользователю и на которую будут ориентироваться
пользователи при работе с приложением.\par
Тогда как атрибут \texttt{android:versionCode} представляет номер версии для
внутреннего использования. Этот номер только определяет, что одна версия
приложения более новая, чем какая-то другая с меньшим номером номером
версии. Этот номер не отображается пользователям.\par
При желании мы также можем определить версию в ресурсах, а здесь
ссылаться на ресурс.

\section{Установка версии SDK}
Для управления версией android sdk в файле манифеста определяется элемент
\texttt{<uses-sdk>}. Он может использовать следующие атрибуты:

\begin{itemize}
	\item \texttt{minSdkVersion}: минимальная поддерживаемая версия SDK;
	\item \texttt{targetSdkVersion}: оптимальная версия;
	\item \texttt{maxSdkVersion}: максимальная версия.
\end{itemize}

Версия определяется номером API, например, \texttt{Jelly Beans 4.1}
имеет версию 16, а \texttt{Android 11} имеет версию 30.\par
Пример использования показан в листнинге~\ref{lst:sdk}.

\begin{lstlisting}[language=XML
	, caption=\leftline{}
	, label=lst:sdk
	]
<manifest
    xmlns:android="http://schemas.android.com/apk/res/android"
    xmlns:tools="http://schemas.android.com/tools"
    android:versionName="1.0"
    android:versionCode="1">

    <uses-sdk android:minSdkVersion="22" android:targetSdkVersion="33" />

	...

</manifest>
\end{lstlisting}

\section{Установка разрешений}
Иногда приложению требуются разрешения на доступ к определенным
ресурсам, например, к списку контактов, камере и т.д. Чтобы приложение
могло работать с тем же списком контактов, в файле манифесте необходимо
установить соответствующие разрешения. Для установки разрешений
применяется элемент \texttt{<uses-permission>}.\par
Пример использования показан в листнинге~\ref{lst:permission}.

\begin{lstlisting}[language=XML
	, caption=\leftline{}
	, label=lst:permission
	]
<manifest
    xmlns:android="http://schemas.android.com/apk/res/android"
    xmlns:tools="http://schemas.android.com/tools"
    android:versionName="1.0"
    android:versionCode="1">

    <uses-sdk android:minSdkVersion="22" android:targetSdkVersion="33" />
    <uses-permission android:name="android.permission.READ_CONTACTS" />

	...

</manifest>
\end{lstlisting}

Атрибут \texttt{android:name} устанавливает название разрешения: в данном
случае на чтение списка контактов и использование камеры. Опционально можно
установить максимальную версию sdk посредством атрибута
\texttt{android:maxSdkVersion}, который принимает номер API.

\section{Поддержка разных разрешений}
Мир устройств Android очень сильно фрагментирован, здесь встречаются как
гаджеты с небольшим экраном, так и большие широкоэкранные телевизоры.
И бывают случаи, когда надо ограничить использование приложения для
определенных разрешений экранов. Для этого в файле манифеста
определяется элемент \texttt{<supports-screens>}.\par
Пример использования показан в листнинге~\ref{}.

\begin{lstlisting}[language=XML
	, caption=\leftline{}
	, label=
	]
<manifest
    xmlns:android="http://schemas.android.com/apk/res/android"
    xmlns:tools="http://schemas.android.com/tools">

	<supports-screens
	 android:largeScreens="true"
	 android:normalScreens="true"
	 android:smallScreens="false"
	 android:xlargeScreens="true" />

	...

</manifest>
\end{lstlisting}

Данный элемент принимает четыре атрибута:

\begin{itemize}
	\item \texttt{android:largeScreens} --- экраны с диагональю от 4.5 до 10"
	\item \texttt{android:normalScreens} --- экраны с диагональю от 3 до 4.5"
	\item \texttt{android:smallScreens} --- экраны с диагональю меньше 3"
	\item \texttt{android:xlargeScreens} --- экраны с диагональю больше 10"
\end{itemize}

Если атрибут имеет значение true, то приложение будет поддерживаться
соответствующим размером экрана.

\section{Запрет на изменение ориентации}
Приложение в зависимости от положения гаджета может находиться в
альбомной и портретной ориентации. Не всегда это бывает удобно. Мы
можем сделать, чтобы приложение вне зависимости от поворота гаджета
использовало только одну ориентацию. Для этого в файле манифеста у
требуемой activity надо установить атрибут android:screenOrientation.
Например, запретим альбомную ориентацию.\par
Пример использования показан в листнинге~\ref{lst:orientation}.

\begin{lstlisting}[language=XML
	, caption=\leftline{}
	, label=lst:orientation
	]
<activity
	android:name=".MainActivity"
	android:exported="true"
	android:screenOrientation="portrait">
	<intent-filter>
		<action android:name="android.intent.action.MAIN" />

		<category android:name="android.intent.category.LAUNCHER" />
	</intent-filter>

	<meta-data
		android:name="android.app.lib_name"
		android:value="" />
</activity>
\end{lstlisting}

Значение \texttt{android:screenOrientation="portrait"} указывает,
что данная activity будет находиться только в портретной ориентации.
Если же надо установить только альбомную ориентацию, тогда надо использовать
значение \texttt{android:screenOrientation="landscape"}.

\section{Переход между активити}
Для взаимодействия между различными объектами activity ключевым
классом является \texttt{android.content.Intent}. Он представляет собой
задачу, которую надо выполнить приложению.\par
Чтобы из MainActivity запустить SecondActivity, надо вызвать метод
\texttt{startActivity()}:

\begin{lstlisting}[language=Java
	, caption=\leftline{Создание намеренья}
	]
Intent intent = new Intent(this, SecondActivity.class);
startActivity(intent);
\end{lstlisting}

В качестве параметра в метод \texttt{startActivity} передается объект Intent.
Для своего создания Intent в конструкторе принимает два параметра: контекст
выполнения (в данном случае это текущий объект MainActivity) и класс,
который используется объектом Intent и представляет передаваемые в задачу
данные (фактически класс activity, которую мы будем запускать).\par
Теперь рассмотрим реализацию перехода от одной Activity к другой.
Пример показан в листнинге~\ref{lst:intent}.

\begin{lstlisting}[language=Java
	, caption=\leftline{}
	, label=lst:intent
	]
Button button = findViewById(R.id.button);
button.setOnClickListener(new View.OnClickListener() {
	@Override
	public void onClick(View v) {
		Intent intent = new Intent(MainActivity.this, SecondActivity.class);
		startActivity(intent);
	}
});
\end{lstlisting}

\section{Передача данных между Activity}
Для передачи данных применяется метод \texttt{putExtra()}, который в качестве
значения позволяет передать данные простейших типов - String, int, float,
double, long, short, byte, char, массивы этих типов, либо объект интерфейса
Serializable.\par
Чтобы получить отправленные данные при загрузке SecondActivity, можно
воспользоваться методом \texttt{get()}, в который передается ключ объекта:

\begin{lstlisting}[language=Java
	, caption=\leftline{Получение намеренья}
	]
Bundle arguments = getIntent().getExtras();
String name = arguments.get("hello").toString();
\end{lstlisting}

В зависимости от типа отправляемых данных при их получении мы можем
использовать ряд методов объекта Bundle. Все они в качестве параметра
принимают ключ объекта. Основные из них:
\begin{itemize}
	\item \texttt{get()}: универсальный метод, который возвращает значение типа
		Object. Соответственно поле получения данное значение необходимо
		преобразовать к нужному типу
	\item \texttt{getString()}: возвращает объект типа String;
	\item \texttt{getInt()}: возвращает значение типа int;
	\item \texttt{getByte()}: возвращает значение типа byte;
	\item \texttt{getChar()}: возвращает значение типа char;
	\item \texttt{getShort()}: возвращает значение типа short;
	\item \texttt{getLong()}: возвращает значение типа long;
	\item \texttt{getFloat()}: возвращает значение типа float;
	\item \texttt{getDouble()}: возвращает значение типа double;
	\item \texttt{getBoolean()}: возвращает значение типа boolean;
	\item \texttt{getCharArray()}: возвращает массив объектов char;
	\item \texttt{getIntArray()}: возвращает массив объектов int;
	\item \texttt{getFloatArray()}: возвращает массив объектов float;
	\item \texttt{getSerializable()}: возвращает объект
		интерфейса Serializable.
\end{itemize}

Пример использования показан в
листнингах~\ref{lst:intent:put}\,-\,\ref{lst:intent:get}..

\begin{lstlisting}[language=Java
	, caption=\leftline{}
	, label=lst:intent:put
	]
Button button = findViewById(R.id.button);
button.setOnClickListener(new View.OnClickListener() {
	@Override
	public void onClick(View v) {
		Intent intent = new Intent(MainActivity.this, SecondActivity.class);

		EditText editText = findViewById(R.id.input);
		intent.putExtra("msg", editText.getText().toString());
		
		startActivity(intent);
	}
});
\end{lstlisting}

\begin{lstlisting}[language=Java
	, caption=\leftline{}
	, label=lst:intent:get
	]
Bundle arguments = getIntent().getExtras();
String msg = arguments.get("msg").toString();

TextView textView = findViewById(R.id.output);
textView.setText(msg);
\end{lstlisting}

\section{Передача сложных объектов}
В Android можно передавать передавать сложные объекты.
В этом случае используется механизм сериализации.\par
Для получения данных применяется метод \texttt{getSerializable()},
поскольку класс Msg (Листнинг~\ref{lst:model:msg})
реализует интерфейс Serializable.
Таким образом, мы можем передать один единственый объект вместо набора
разрозненных данных.\par
Пример использования показан в листнинге~\ref{lst:serializable}.

\begin{lstlisting}[language=Java
	, caption=\leftline{}
	, label=lst:model:msg
	]
public class Msg implements Serializable {
	private String message;

	public Msg(String message) {
		this.message = message;
	}

	public String getMessage() {
		return message;
	}

	public void setMessage(String message) {
		this.message = message;
	}
}
\end{lstlisting}

\begin{lstlisting}[language=Java
	, caption=\leftline{}
	, label=lst:serializable
	]
Button button = findViewById(R.id.button);
button.setOnClickListener(new View.OnClickListener() {
	@Override
	public void onClick(View v) {
		Intent intent = new Intent(MainActivity.this, SecondActivity.class);

		Msg msg = new Msg(editText.getText().toString());
		intent.putExtra("model:msg", msg);

		startActivity(intent);
	}
});
\end{lstlisting}

\section{Parcelable}
Возможность сериализации объектов предоставляется напрямую
инфраструктурой языка Java. Однако Android также предоставляет
интерфейс Parcelable, который, по сути, также позволяет сериализовать
объекты, как и Serializable, но является более оптимизированным для
Android. И подобные объекты Parcelable также можно передавать между
двумя activity или использовать каким-то иным образом
(Листнинг~\ref{lst:model:parcelable}).

\begin{lstlisting}[language=Java
	, caption=\leftline{}
	, label=lst:model:parcelable
	]
public class MsgP implements Parcelable {
	private String message;

	public String getMessage() {
		return message;
	}

	public void setMessage(String message) {
		this.message = message;
	}

	protected MsgP(Parcel in) {
		message = in.readString();
	}

	public static final Creator<MsgP> CREATOR = new Creator<MsgP>() {
		@Override
		public MsgP createFromParcel(Parcel in) {
			return new MsgP(in);
		}

		@Override
		public MsgP[] newArray(int size) {
			return new MsgP[size];
		}
	};

	@Override
	public int describeContents() {
		return 0;
	}

	@Override
	public void writeToParcel(@NonNull Parcel dest, int flags) {
		dest.writeString(message);
	}
}
\end{lstlisting}

Интерфейс android.os.Parcelable предполагает реализацию двух методов:
\texttt{describeContents()} и \texttt{writeToParcel()}.
Первый метод описывает контент и возвращает некторое числовое значение.
Второй метод пишет в объект Parcel содержимое объекта MsgP.
Для записи данных объекта в Parcel используется ряд методов, каждый из
которых предназначен для определенного типа данных. Основные методы:
\begin{itemize}
	\item \texttt{writeString()};
	\item \texttt{writeInt()};
	\item \texttt{writeFloat()};
	\item \texttt{writeDouble()};
	\item \texttt{writeByte()};
	\item \texttt{writeLong()};
	\item \texttt{writeIntArray()};
	\item \texttt{writeValue()} (записывает объект типа Object);
	\item \texttt{writeParcelable()} (записывает объект типа Parcelable).
\end{itemize}

Кроме того, объект Parcelable должен содержать статическое поле
CREATOR, которое представляет объект Creator<MsgP>. Причем этот объект
реализует два метода. Они нужны для создания их ранее сериализованных
данных исходных объектов типа MsgP.\par
Так, метод \texttt{newArray()} создает массив объект MsgP.
Метод \texttt{createFromParcel} создает из Parcel новый объект типа MsgP.
То есть этот метод противоположен по действию методу \texttt{writeToParcel}.
Для получения данных из Parcel применяются методы типа \texttt{readString()},
\texttt{readInt()}, \texttt{readParcelable()} и так далее --- для чтения
определенных типов данных.\par
Причем важно, что данные в \texttt{createFromParcel} считываются из объекта
Parcel именно в том порядке, в котором они добавляются в этот объект в методе
\texttt{writeToParcel}.\par
Пример использования показан в листнинге~\ref{lst:parcelable}.

\begin{lstlisting}[language=Java
	, caption=\leftline{}
	, label=lst:parcelable
	]
Bundle arguments = getIntent().getExtras();
MsgP msgP = arguments.getParcelable("model:msg:p");

TextView textView = findViewById(R.id.output);
textView.setText(msgP.getMessage());
\end{lstlisting}

\section{Получение результата из Activity}
Мы можем не только передавать данные запускаемой activity, но и ожидать
от нее некоторого результата работы.\par
Для получения результата работы запускаемой activity
необходимо использовать \textbf{Activity Result API}.\par
Activity Result API предоставляет компоненты для регистрации, запуска и
обработки результата другой Activity. Одним из преимуществ применения
Activity Result API является то, что он отвязывает результат Activity от самой
Activity. Это позволяет получить и обработать результат, даже если Activity,
которая возвращает результат, в силу ограничений памяти или в силу других
причин завершила свою работу.\par

\subsection{Регистрация функции для получения результата}
Для регистрации функции, которая будет обрабатывать результат, Activity
Result API предоставляет метод \texttt{registerForActivityResult()}.
Этот метод в качестве параметров принимает объекты
\texttt{ActivityResultContract} и \texttt{ActivityResultCallback}
и возвращает объект \texttt{ActivityResultLauncher},
который применяется для запуска другой activity.\par
\texttt{ActivityResultContract} определяет контракт: данные какого типа будут
подаваться на вход и какой тип будет представлять результат.\par
\texttt{ActivityResultCallback} представляет интерфейс с единственным методом
\texttt{onActivityResult()}, который определяет обработку полученного
результата. Когда вторая activity закончит работу и возвратит результат,
то будет как раз вызываться этот метод.
Результат передается в метод в качестве параметра.
При этом тип параметра должен соответствовать типу результата,
определенного в \texttt{ActivityResultContract}.

\subsection{Запуск activity для получения результата}
Метод \texttt{registerForActivityResult()} регистрирует функцию-колбек
и возвращает объект \texttt{ActivityResultLauncher}.
С помощью этого мы можем запустить activity.
Для этого у объекта \texttt{ActivityResultLauncher} вызывается метод
\texttt{launch()}.\par
В метод \texttt{launch()} передается объект того типа, который определен
объектом \texttt{ActivityResultContracts} в качестве входного.

\subsection{Практическое применение Activity Result API}
Пример использования показан в листнинге~\ref{lst:activity_result}.

\begin{lstlisting}[language=Java
	, caption=\leftline{}
	, label=lst:activity_result
	]
public class ThirdActivity extends AppCompatActivity {
	static final String AGE_KEY = "AGE";
	static final String ACCESS_MESSAGE="ACCESS_MESSAGE";

	ActivityResultLauncher<Intent> mStartForResult =
			registerForActivityResult(new ActivityResultContracts.StartActivityForResult(),
					new ActivityResultCallback<ActivityResult>() {
						@Override
						public void onActivityResult(ActivityResult result) {
							TextView textView = findViewById(R.id.textView);
							if(result.getResultCode() == Activity.RESULT_OK){
								Intent intent = result.getData();
								String accessMessage =
										intent.getStringExtra(ACCESS_MESSAGE);
								textView.setText(accessMessage);
							}
							else{
								textView.setText("Error");
							}
						}
					});

	@Override
	protected void onCreate(Bundle savedInstanceState) {
		super.onCreate(savedInstanceState);
		setContentView(R.layout.activity_third);
	}

	public void onClick(View view) {
		EditText ageBox = findViewById(R.id.age);
		String age = ageBox.getText().toString();
		Intent intent = new Intent(this, MainActivity.class);
		intent.putExtra(AGE_KEY, age);
		mStartForResult.launch(intent);
	}
}
\end{lstlisting}

\section{Взаимодействие между Activity}
В прошлых темах мы рассмотрели жизненный цикл activity и запуск новых
activity с помощью объекта Intent. Теперь рассмотрим некоторые
особенности взаимодействия между четырьмя activity в одном приложении.

\subsection{Управление стеком activity}
Для управления стеком из activity Android предлагает нам использовать
флаги --- константы, определенные в классе Intent. Применение
определенного флага позволит нам определенным образом изменить
положение в стеке для определенных activity.\par
Пример использования показан в листнинге~\ref{lst:intent:stack}.

\begin{lstlisting}[language=Java
	, caption=\leftline{}
	, label=lst:intent:stack
	]
Button button = findViewById(R.id.button);
button.setOnClickListener(new View.OnClickListener() {
	@Override
	public void onClick(View v) {
		Intent intent = new Intent(SecondActivity.this, MainActivity.class);
		intent.addFlags(Intent.FLAG_ACTIVITY_CLEAR_TOP |
				Intent.FLAG_ACTIVITY_SINGLE_TOP);
		startActivity(intent);
	}
});
\end{lstlisting}

\clearpage

\section*{\LARGE{Вывод}}
\addcontentsline{toc}{section}{Вывод}
В этой практической работе были получены знания о содержимом файла
манифеста и управления им. Такие как: установка версии и разрешений или
запрет на изменение ориентации. Еще научились осуществлять переходы
между Activity, передавать различные данные между ними, в том числе
объектов.

