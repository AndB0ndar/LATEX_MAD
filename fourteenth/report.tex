\graphicspath{{./fourteenth/img}}

\section*{\LARGE{Введение}}
\addcontentsline{toc}{section}{Введение}

\textbf{Задание}:
\begin{itemize}
	\item Реализовать небольшой проект работы с контактами. Добавление
		контактов.
	\item Реализовать создание провайдера контента, чтобы другие
		приложения могли обращаться к данным нашего приложения через
		некоторый API. Определение контракта через класс FriendsContract.
	\item Реализовать получение данных в провайдере используя метод
		query().
	\item Реализовать добавление данных, применяя метод insert()
	\item Реализовать удаление данных, применяя метод delete()
	\item Реализовать обновление данных, применяя метод update()
	\item Для работы провайдера контента, реализовать внесение
		соответствующих изменений в файл AndroidManifest.xml
	\item Реализовать применение провайдера контента. Получение данных.
		Добавление данных. Обновление данных. Удаление данных.
	\item Реализовать асинхронную загрузку данных
	\item Реализовать пример работы с json.
	\item Реализовать пример работы с XML
\end{itemize}

\clearpage

\section*{\LARGE{Выполнение практической работы}}
\addcontentsline{toc}{section}{Выполнение практической работы}
Наше приложение может сохранять разнообразную информацию о
пользователе, какие-то связанные данные в файлах или настройках. Однако
ОС Android уже хранит ряд важной информации, связанной с пользователем,
к которой имеем доступ и которую мы можем использовать. Это и списки
контактов, и файлы сохраненных изображений и видеоматериалов, и какието отметки о звонках и т.д., то есть некоторый контент. А для доступа к
этому контенту в OC Android определены провайдеры контента (content
provider).\par
В Android имеются следующие встроенные провайдеры, определенные в
пакете android.content:

\begin{itemize}
	\item AlarmClock: управление будильником;
	\item Browser: история браузера и закладки;
	\item CalendarContract: каледарь и информаци о событиях;
	\item CallLog: информация о звонках;
	\item ContactsContract: контакты;
	\item MediaStore: медиа-файлы;
	\item SearchRecentSuggestions: подсказки по поиску;
	\item Settings: системные настройки;
	\item UserDictionary: словарь слов, которые используются
		для быстрого набора;
	\item VoicemailContract: записи голосовой почты.
\end{itemize}

\section{Работа с контактами}
Контакты в Android обладают встроенным API, который позволяет получать
и изменять список контактов. Все контакты хранятся в базе данных SQLite,
однако они не представляют единой таблицы. Для контактов отведено три
таблицы, связанных отношением один-ко-многим: таблица для хранения
информации о людях, таблица их телефонов и и таблица адресов их
электронных почт. Но благодаря Android API мы можем абстрагироваться от
связей между таблицами.\par
Продолжим работу с проектом из прошлой темы и добавим в него
возможность добавления новых контактов. Добавление контактов
представляет собой запрос на изменение списка контактов, то есть его
запись. Поэтому нам надо установить соответствующее разрешение в файле
манифеста. Возьмем проект из прошлого материала и добавим в него в файл
AndroidManifest.xml разрешение android.permission.WRITE\_CONTACTS:
Итак, для доступа к контактам нам надо установить разрешение
\texttt{android.permission.READ\_CONTACTS} в файле манифеста приложения.\par
Для вывода списка контактов в файле разметки определим
следующую разметку интерфейса:

\begin{lstlisting}[language=XML
	, label=lst:
	]
<?xml version="1.0" encoding="utf-8"?>
<androidx.constraintlayout.widget.ConstraintLayout
    xmlns:android="http://schemas.android.com/apk/res/android"
    xmlns:app="http://schemas.android.com/apk/res-auto"
    android:layout_width="match_parent"
    android:layout_height="match_parent">
    <EditText
        android:id="@+id/newContact"
        android:layout_width="0dp"
        android:layout_height="wrap_content"
        app:layout_constraintBottom_toTopOf="@id/header"
        app:layout_constraintLeft_toLeftOf="parent"
        app:layout_constraintRight_toLeftOf="@id/addBtn"
        app:layout_constraintTop_toTopOf="parent" />
    <Button
        android:id="@+id/addBtn"
        android:text="Add"
        android:layout_width="wrap_content"
        android:layout_height="wrap_content"
        android:onClick="onAddContact"
        app:layout_constraintBottom_toTopOf="@id/header"
        app:layout_constraintLeft_toRightOf="@id/newContact"
        app:layout_constraintRight_toRightOf="parent"
        app:layout_constraintTop_toTopOf="parent" />
    <TextView
        android:id="@+id/header"
        android:layout_width="0dp"
        android:layout_height="wrap_content"
        android:text="Контакты"
        android:textSize="18sp"
        app:layout_constraintBottom_toTopOf="@id/contactList"
        app:layout_constraintLeft_toLeftOf="parent"
        app:layout_constraintRight_toRightOf="parent"
        app:layout_constraintTop_toBottomOf="@id/newContact" />
    <ListView
        android:id="@+id/contactList"
        android:layout_width="0dp"
        android:layout_height="0dp"
        app:layout_constraintBottom_toBottomOf="parent"
        app:layout_constraintLeft_toLeftOf="parent"
        app:layout_constraintRight_toRightOf="parent"
        app:layout_constraintTop_toBottomOf="@+id/header" />
</androidx.constraintlayout.widget.ConstraintLayout>
\end{lstlisting}

Для вывода списка контактов воспользуемся элементом ListView. И в классе
MainActivity получим контакты:

\begin{lstlisting}[language=Java
	, label=lst:
	]
	private static final int REQUEST_CODE_READ_CONTACTS = 1;
	private static boolean READ_CONTACTS_GRANTED = false;
	ArrayList<String> contacts = new ArrayList<>();
	Button addBtn;

	@Override
	protected void onCreate(Bundle savedInstanceState) {
		super.onCreate(savedInstanceState);
		setContentView(R.layout.activity_contact);
		addBtn = findViewById(R.id.addBtn);
		// получаем разрешения
		int hasReadContactPermission = ContextCompat.checkSelfPermission(this,
				android.Manifest.permission.READ_CONTACTS);
		// если устройство до API 23, устанавливаем разрешение
		if (hasReadContactPermission ==
				PackageManager.PERMISSION_GRANTED) {
			READ_CONTACTS_GRANTED = true;
		} else {
			// вызываем диалоговое окно для установки разрешений
			ActivityCompat.requestPermissions(this, new
							String[]{android.Manifest.permission.READ_CONTACTS,
							android.Manifest.permission.WRITE_CONTACTS},
					REQUEST_CODE_READ_CONTACTS);
		}
		// если разрешение установлено, загружаем контакты
		if (READ_CONTACTS_GRANTED) {
			loadContacts();
		}
		addBtn.setEnabled(READ_CONTACTS_GRANTED);
	}

	@Override
	public void onRequestPermissionsResult(int requestCode, @NonNull String[]
			permissions, @NonNull int[] grantResults) {
		super.onRequestPermissionsResult(requestCode, permissions, grantResults);

		if (requestCode == REQUEST_CODE_READ_CONTACTS) {
			if (grantResults.length > 0 && grantResults[0] ==
					PackageManager.PERMISSION_GRANTED) {
				READ_CONTACTS_GRANTED = true;
			}
			addBtn.setEnabled(READ_CONTACTS_GRANTED);
		}
		if (READ_CONTACTS_GRANTED) {
			loadContacts();
		} else {
			Toast.makeText(this, "Требуется установить разрешения",
					Toast.LENGTH_LONG).show();
		}
	}

	public void onAddContact(View v) {
		ContentValues contactValues = new ContentValues();
		EditText contactText = findViewById(R.id.newContact);
		String newContact = contactText.getText().toString();
		contactText.setText("");
		contactValues.put(ContactsContract.RawContacts.ACCOUNT_NAME, newContact);
		contactValues.put(ContactsContract.RawContacts.ACCOUNT_TYPE, newContact);
		Uri newUri = getContentResolver().insert(ContactsContract.RawContacts.CONTENT_URI, contactValues);
		long rawContactsId = ContentUris.parseId(newUri);
		contactValues.clear();
		contactValues.put(ContactsContract.Data.RAW_CONTACT_ID
				,	rawContactsId);
		contactValues.put(ContactsContract.Data.MIMETYPE
				, ContactsContract.CommonDataKinds.StructuredName.CONTENT_ITEM_TYPE);

		contactValues.put(ContactsContract.CommonDataKinds.StructuredName.DISPLAY_NAME, newContact);
		getContentResolver().insert(ContactsContract.Data.CONTENT_URI, contactValues);
		Toast.makeText(getApplicationContext(), newContact + " добавлен в список контактов", Toast.LENGTH_LONG).show();
		loadContacts();
	}

	private void loadContacts() {
		contacts.clear();
		ContentResolver contentResolver = getContentResolver();
		Cursor cursor =
				contentResolver.query(ContactsContract.Contacts.CONTENT_URI, null, null,
						null, null);
		if (cursor != null) {
			while (cursor.moveToNext()) {
				// получаем каждый контакт
				int id = cursor.getColumnIndex(ContactsContract.Contacts.DISPLAY_NAME_PRIMARY);
				String contact = cursor.getString(id);
				// добавляем контакт в список
				contacts.add(contact);
			}
			cursor.close();
		}
		// создаем адаптер
		ArrayAdapter<String> adapter = new ArrayAdapter<>(this,
				android.R.layout.simple_list_item_1, contacts);
		// устанавливаем для списка адаптер
		ListView contactList = findViewById(R.id.contactList);
		contactList.setAdapter(adapter);
	}
\end{lstlisting}

\section{Создание провайдера контента}
\subsection{Определение контракта}
Контент-провайдеры (content providers) позволяют обращаться одним
приложениям к данным других приложений. И мы таже можем сделать,
чтобы другие приложения могули обращаться к данным нашего приложения
через некоторый API. Для этого нам надо создать свой контент-провайдер.\par
Вначале добавим в проект класс FriendsContract, который будет описывать
основные значения, столбцы, адреса uri, используемые в контент-провайдере:

\begin{lstlisting}[language=Java
	, label=lst:
	]
import android.content.ContentUris;
import android.net.Uri;

public class FriendsContract {
	static final String TABLE_NAME = "friends";
	static final String CONTENT_AUTHORITY = "com.example.fourteenth";
	static final Uri CONTENT_AUTHORITY_URI = Uri.parse("content://" +
			CONTENT_AUTHORITY);
	static final String CONTENT_TYPE = "vnd.android.cursor.dir/vnd." +
			CONTENT_AUTHORITY + "." + TABLE_NAME;
	static final String CONTENT_ITEM_TYPE= "vnd.android.cursor.item/vnd." +
			CONTENT_AUTHORITY + "." + TABLE_NAME;

	public static class Columns{
		public static final String _ID = "_id";
		public static final String NAME = "Name";
		public static final String EMAIL = "Email";
		public static final String PHONE = "Phone";
		private Columns(){
		}
	}
	static final Uri CONTENT_URI =
			Uri.withAppendedPath(CONTENT_AUTHORITY_URI, TABLE_NAME);
	// создает uri с помощью id
	static Uri buildFriendUri(long taskId){
		return ContentUris.withAppendedId(CONTENT_URI, taskId);
	}
	// получает id из uri
	static long getFriendId(Uri uri){
		return ContentUris.parseId(uri);
	}
}
\end{lstlisting}

Далее добавим в проект новый класс AppDatabase:

\begin{lstlisting}[language=Java
	, label=lst:
	]
package com.example.fourteenth;

import android.content.Context;
import android.database.sqlite.SQLiteDatabase;
import android.database.sqlite.SQLiteOpenHelper;

public class AppDatabase extends SQLiteOpenHelper {
	public static final String DATABASE_NAME = "friends.db";
	public static final int DATABASE_VERSION = 1;
	private static AppDatabase instance = null;

	private AppDatabase(Context context){
		super(context, DATABASE_NAME, null, DATABASE_VERSION);
	}

	static AppDatabase getInstance(Context context){
		if(instance == null){
			instance = new AppDatabase(context);
		}
		return instance;
	}

	@Override
	public void onCreate(SQLiteDatabase db) {
		String sql = "CREATE TABLE " + FriendsContract.TABLE_NAME + "(" +
				FriendsContract.Columns._ID + " INTEGER PRIMARY KEY NOT NULL, " +
		FriendsContract.Columns.NAME + " TEXT NOT NULL, " +
				FriendsContract.Columns.EMAIL + " TEXT, " +
				FriendsContract.Columns.PHONE + " TEXT NOT NULL)";
		db.execSQL(sql);
		// добавление начальных данных
		db.execSQL("INSERT INTO "+ FriendsContract.TABLE_NAME +" (" +
				FriendsContract.Columns.NAME
				+ ", " + FriendsContract.Columns.PHONE + ") VALUES ('Tom', '+12345678990');");
		db.execSQL("INSERT INTO "+ FriendsContract.TABLE_NAME +" (" +
				FriendsContract.Columns.NAME
				+ ", " + FriendsContract.Columns.EMAIL + ", " +
				FriendsContract.Columns.PHONE +
				" ) VALUES ('Bob', 'bob@gmail.com', '+13456789102');");
	}
	@Override
	public void onUpgrade(SQLiteDatabase db, int oldVersion, int newVersion) {
	}
}
\end{lstlisting}

Данный класс по принципу синглтона организует доступ к базе данных и,
кроме того, создает саму базу данных и добавляет в нее начальные данные.

\subsection{Создание провайдера}
Продолжим рабоу с проектом из прошлого материала и добавим в него класс
AppProvider, который, собственно, и будет представлять провайдер
контента:

\begin{lstlisting}[language=Java
	, label=lst:
	]
package com.example.fourteenth;

import android.content.ContentProvider;
import android.content.ContentValues;
import android.content.UriMatcher;
import android.database.Cursor;
import android.database.SQLException;
import android.database.sqlite.SQLiteDatabase;
import android.database.sqlite.SQLiteQueryBuilder;
import android.net.Uri;
import androidx.annotation.NonNull;
import androidx.annotation.Nullable;

public class AppProvider extends ContentProvider {
	private AppDatabase mOpenHelper;
	private static final UriMatcher sUriMatcher = buildUriMatcher();
	public static final int FRIENDS = 100;
	public static final int FRIENDS_ID = 101;
	private static UriMatcher buildUriMatcher(){
		final UriMatcher matcher = new UriMatcher(UriMatcher.NO_MATCH);
		// content://com.example.friendsprovider/FRIENDS
		matcher.addURI(FriendsContract.CONTENT_AUTHORITY,
				FriendsContract.TABLE_NAME, FRIENDS);
		// content://com.example.friendsprovider/FRIENDS/8
		matcher.addURI(FriendsContract.CONTENT_AUTHORITY,
				FriendsContract.TABLE_NAME + "/#", FRIENDS_ID);
		return matcher;
	}
	@Override
	public boolean onCreate() {
		mOpenHelper = AppDatabase.getInstance(getContext());
		return true;
	}
	@Nullable
	@Override
	public Cursor query(@NonNull Uri uri, @Nullable String[] projection,
						@Nullable String selection, @Nullable String[] selectionArgs, @Nullable String
								sortOrder) {
		final int match = sUriMatcher.match(uri);
		SQLiteQueryBuilder queryBuilder = new SQLiteQueryBuilder();
		switch(match){
			case FRIENDS:
				queryBuilder.setTables(FriendsContract.TABLE_NAME);
				break;
			case FRIENDS_ID:
				queryBuilder.setTables(FriendsContract.TABLE_NAME);
				long taskId = FriendsContract.getFriendId(uri);
				queryBuilder.appendWhere(FriendsContract.Columns._ID + " = " +
						taskId);
				break;
			default:
				throw new IllegalArgumentException("Unknown URI: "+ uri);
		}
		SQLiteDatabase db = mOpenHelper.getReadableDatabase();
		return queryBuilder.query(db, projection, selection, selectionArgs, null, null,
				sortOrder);
	}
	@Nullable
	@Override
	public String getType(@NonNull Uri uri) {
		final int match = sUriMatcher.match(uri);
		switch(match){
			case FRIENDS:
				return FriendsContract.CONTENT_TYPE;
			case FRIENDS_ID:
				return FriendsContract.CONTENT_ITEM_TYPE;
			default:
				throw new IllegalArgumentException("Unknown URI: "+ uri);
		}
	}
	@Nullable
	@Override
	public Uri insert(@NonNull Uri uri, @Nullable ContentValues values) {
		final int match = sUriMatcher.match(uri);
		final SQLiteDatabase db;
		Uri returnUri;
		long recordId;
		if (match == FRIENDS) {
			db = mOpenHelper.getWritableDatabase();
			recordId = db.insert(FriendsContract.TABLE_NAME, null, values);
			if (recordId > 0) {
				returnUri = FriendsContract.buildFriendUri(recordId);
			} else {
				throw new SQLException("Failed to insert: " + uri.toString());
			}
		} else {
			throw new IllegalArgumentException("Unknown URI: " + uri);
		}
		return returnUri;
	}
	
	@Override
	public int delete(@NonNull Uri uri, @Nullable String selection, @Nullable
			String[] selectionArgs) {
		final int match = sUriMatcher.match(uri);
		final SQLiteDatabase db = mOpenHelper.getWritableDatabase();
		String selectionCriteria = selection;
		if (match != FRIENDS && match != FRIENDS_ID)
			throw new IllegalArgumentException("Unknown URI: "+ uri);
		if (match==FRIENDS_ID) {
			long taskId = FriendsContract.getFriendId(uri);
			selectionCriteria = FriendsContract.Columns._ID + " = " + taskId;
			if ((selection != null) && (selection.length() > 0)) {
				selectionCriteria += " AND (" + selection + ")";
			}
		}
		return db.delete(FriendsContract.TABLE_NAME, selectionCriteria,
				selectionArgs);
	}
	@Override
	public int update(@NonNull Uri uri, @Nullable ContentValues values,
					  @Nullable String selection, @Nullable String[] selectionArgs) {
		final int match = sUriMatcher.match(uri);
		final SQLiteDatabase db = mOpenHelper.getWritableDatabase();
		String selectionCriteria = selection;
		if(match != FRIENDS && match != FRIENDS_ID)
			throw new IllegalArgumentException("Unknown URI: "+ uri);
		if(match==FRIENDS_ID) {
			long taskId = FriendsContract.getFriendId(uri);
			selectionCriteria = FriendsContract.Columns._ID + " = " + taskId;
			if ((selection != null) && (selection.length() > 0)) {
				selectionCriteria += " AND (" + selection + ")";
			}
		}
		return db.update(FriendsContract.TABLE_NAME, values, selectionCriteria,
				selectionArgs);
	}
}
\end{lstlisting}

Класс провайдера контента должен наследоваться от абстрактного класса
ContentProvider, который определяет ряд методов для работы с данными, в
частности, методы oncreate, query, insert, update, delete, getType.\par
Но чтобы провайдер контента заработал, необходимо внести изменения в
файл AndroidManifest.xml.
В конец элемента <application> добавим определение провайдера:

\begin{lstlisting}[language=XML
	, label=lst:
	]
<provider
	android:name=".AppProvider"
	android:authorities="com.example.fourteenth"
	android:exported="false" />
\end{lstlisting}

В элементе provider атрибут android:authorities указывает на название
провайдера - в данном случае это название, которое определено в прошлой
теме в константе CONTENT\_AUTHORITY в классе FriendsContract.

\subsection{Применение}
В прошлом материале был определен провайдер контента. Рассмотрим, как
его использовать. Сначала определим простейший визуальный интерфейс
для тестирования возможностей провайдера в файле разметки:

\begin{lstlisting}[language=XML
	, label=lst:
	]
<?xml version="1.0" encoding="utf-8"?>
<androidx.constraintlayout.widget.ConstraintLayout xmlns:android="http://schemas.android.com/apk/res/android"
    xmlns:app="http://schemas.android.com/apk/res-auto"
    android:layout_width="match_parent"
    android:layout_height="match_parent"
    xmlns:tools="http://schemas.android.com/tools"
    tools:context=".ContentProviderActivity">
    <Button
        android:id="@+id/getButton"
        android:layout_width="0dp"
        android:layout_height="wrap_content"
        android:text="Get"
        android:onClick="getAll"
        app:layout_constraintBottom_toTopOf="@id/addButton"
        app:layout_constraintLeft_toLeftOf="parent"
        app:layout_constraintRight_toRightOf="parent"
        app:layout_constraintTop_toTopOf="parent" />
    <Button
        android:id="@+id/addButton"
        android:layout_width="0dp"
        android:layout_height="wrap_content"
        android:text="Add"
        android:onClick="add"
        app:layout_constraintBottom_toTopOf="@id/updateButton"
        app:layout_constraintLeft_toLeftOf="parent"
        app:layout_constraintRight_toRightOf="parent"
        app:layout_constraintTop_toBottomOf="@id/getButton" />
    <Button
        android:id="@+id/updateButton"
        android:layout_width="0dp"
        android:layout_height="wrap_content"
        android:text="Update"
        android:onClick="update"
        app:layout_constraintBottom_toTopOf="@id/deleteButton"
        app:layout_constraintLeft_toLeftOf="parent"
        app:layout_constraintRight_toRightOf="parent"
        app:layout_constraintTop_toBottomOf="@id/addButton" />
    <Button
        android:id="@+id/deleteButton"
        android:layout_width="0dp"
        android:layout_height="wrap_content"
        android:text="Delete"
        android:onClick="delete"
        app:layout_constraintLeft_toLeftOf="parent"
        app:layout_constraintRight_toRightOf="parent"
        app:layout_constraintTop_toBottomOf="@id/updateButton" />
</androidx.constraintlayout.widget.ConstraintLayout>
\end{lstlisting}

Здесь определен набор кнопок для вывода списка друзей, а также
добавления, обновления и удаления. Каждая кнопка будет вызывать
соответсствующий метод в классе.\par
Теперь изменим код класса. Для упрощения результаты будем
выводить в окне Logcat с помощью метода Log.d():

\begin{lstlisting}[language=Java
	, label=lst:
	]
private static final String TAG = "MainActivity";

@Override
protected void onCreate(Bundle savedInstanceState) {
	super.onCreate(savedInstanceState);
	setContentView(R.layout.activity_content_provider);
}
// получение всех
public void getAll(View view){
	String[] projection = {
			FriendsContract.Columns._ID,
			FriendsContract.Columns.NAME,
			FriendsContract.Columns.EMAIL,
			FriendsContract.Columns.PHONE
	};
	ContentResolver contentResolver = getContentResolver();
	Cursor cursor = contentResolver.query(FriendsContract.CONTENT_URI,
			projection,
			null,
			null,
			FriendsContract.Columns.NAME);
	if(cursor != null){
		Log.d(TAG, "count: " + cursor.getCount());
		// перебор элементов
		while(cursor.moveToNext()){
			for(int i=0; i < cursor.getColumnCount(); i++){
				Log.d(TAG, cursor.getColumnName(i) + " : " + cursor.getString(i));
			}
			Log.d(TAG, "=========================");
		}
		cursor.close();
	}
	else{
		Log.d(TAG, "Cursor is null");
	}
}
// Добавление
public void add(View view){
	ContentResolver contentResolver = getContentResolver();
	ContentValues values = new ContentValues();
	values.put(FriendsContract.Columns.NAME, "Sam");
	values.put(FriendsContract.Columns.EMAIL, "sam@gmail.com");
	values.put(FriendsContract.Columns.PHONE, "+13676254985");
	Uri uri = contentResolver.insert(FriendsContract.CONTENT_URI, values);
	Log.d(TAG, "Friend added");
}
// Обновление
public void update(View view){
	ContentResolver contentResolver = getContentResolver();
	ContentValues values = new ContentValues();
	values.put(FriendsContract.Columns.EMAIL, "sammy@gmail.com");
	values.put(FriendsContract.Columns.PHONE, "+55555555555");
	String selection = FriendsContract.Columns.NAME + " = 'Sam'";
	int count = contentResolver.update(FriendsContract.CONTENT_URI, values,
			selection, null);
	Log.d(TAG, "Friend updated");
}
// Удаление
public void delete(View view){
	ContentResolver contentResolver = getContentResolver();
	String selection = FriendsContract.Columns.NAME + " = ?";
	String[] args = {"Sam"};
	int count = contentResolver.delete(FriendsContract.CONTENT_URI,
			selection, args);
	Log.d(TAG, "Friend deleted");
}
\end{lstlisting}

\subsection{Асинхронная загрузка данных}
В базе данных может быть много данных, и их загрузка может занять
некоторое время. В этом случае можно воспользоваться асинхронной
загрузкой данных. Для этого класс activity или фрагмента должен
реализовать интерфейс LoaderManager.LoaderCallbacks<Cursor>.\par
Возьмем проект из прошлого материала, где реализован провайдер контента
AppProvider, и изменим класс:

\begin{lstlisting}[language=Java
	, label=lst:
	]

public class AsyncContentProviderActivity extends AppCompatActivity implements
		LoaderManager.LoaderCallbacks<Cursor> {

	private static final String TAG = "MainActivity";
	private static final int LOADER_ID = 225;

	@Override
	protected void onCreate(Bundle savedInstanceState) {
		super.onCreate(savedInstanceState);
//		setContentView(R.layout.activity_async_content_provider);
		// запускаем загрузку данных через провайдер контента
		LoaderManager.getInstance(this).initLoader(LOADER_ID, null, this);
	}

	@NonNull
	@Override
	public Loader<Cursor> onCreateLoader(int id, @Nullable Bundle args) {
		String[] projection = {
				FriendsContract.Columns._ID,
				FriendsContract.Columns.NAME,
				FriendsContract.Columns.EMAIL,
				FriendsContract.Columns.PHONE
		};
		if(id == LOADER_ID)
			return new CursorLoader(this, FriendsContract.CONTENT_URI,
					projection,
					null,
					null,
					FriendsContract.Columns.NAME);
		else
			throw new InvalidParameterException("Invalid loader id");
	}

	@Override
	public void onLoadFinished(@NonNull Loader<Cursor> loader, Cursor data) {
		if(data != null){
			Log.d(TAG, "count: " + data.getCount());
			// перебор элементов
			while(data.moveToNext()){
				for(int i=0; i < data.getColumnCount(); i++){
					Log.d(TAG, data.getColumnName(i) + " : " + data.getString(i));
				}
				Log.d(TAG, "=========================");
			}
			data.close();
		}
		else{
			Log.d(TAG, "Cursor is null");
		}
	}

	@Override
	public void onLoaderReset(@NonNull Loader<Cursor> loader) {
		Log.d(TAG, "onLoaderReset...");
	}
}
\end{lstlisting}

Интерфейс LoaderManager.LoaderCallbacks<Cursor> предполагает
реализацию трех методов. Метод onCreateLoader() загружает курсор. В этот
метод в качестве параметров передаются числовой код операции и объект
Bundle. Числовой код передается при запуске загрузки курсора. В данном
случае в качестве такого кода использует константа LOADER\_ID.\par
Чтобы запустить загрузку данных, в методе onCreate вызывается метод
LoaderManager.getInstance(this).initLoader(LOADER\_ID, null, this);.
Первый параметр - числовой код, а второй - объект Bundle. Это те значения,
которые мы можем получить в методе onCreateLoader. И третий - объект
Context.

\section{Работа с json}
Для работы с форматом json нет встроенных средств, но есть куча библиотек
и пакетов, которые можно использовать для данной цели. Одним из наиболее
популярных из них является пакет com.google.code.gson.
Для его использования в проекте Android в файл guild.gradle, который
относится к модулю app, в секцию dependencies необходимо добавить
соответствующую зависимость:

\begin{verbatim}
implementation 'com.google.code.gson:gson:2.8.8'
\end{verbatim}

После добавления пакета в проект добавим новый класс User, который будет
представлять данные:

\begin{lstlisting}[language=Java
	, label=lst:
	]
public class User {

	private String name;
	private int age;

	public User() {
	}

	User(String name, int age){
		this.name = name;
		this.age = age;
	}
	public String getName() {
		return name;
	}
	public void setName(String name) {
		this.name = name;
	}
	public int getAge() {
		return age;
	}
	public void setAge(int age) {
		this.age = age;
	}
	@Override
	public String toString(){
		return "Имя: " + name + " Возраст: " + age;
	}
}
\end{lstlisting}

Объекты этого класса мы будем сериализовать в формат json и наоборот
десериализовать из файла.\par
Для работы с json добавим следующий класс JSONHelper:

\begin{lstlisting}[language=Java
	, label=lst:
	]
class JSONHelper {
	private static final String FILE_NAME = "data.json";
	static boolean exportToJSON(Context context, List<User> dataList) {
		Gson gson = new Gson();
		DataItems dataItems = new DataItems();
		dataItems.setUsers(dataList);
		String jsonString = gson.toJson(dataItems);
		try(FileOutputStream fileOutputStream =
					context.openFileOutput(FILE_NAME, Context.MODE_PRIVATE)) {
			fileOutputStream.write(jsonString.getBytes());
			return true;
		} catch (Exception e) {
			e.printStackTrace();
		}
		return false;
	}
	static List<User> importFromJSON(Context context) {
		try(FileInputStream fileInputStream = context.openFileInput(FILE_NAME);
			InputStreamReader streamReader = new
					InputStreamReader(fileInputStream)){
			Gson gson = new Gson();
			DataItems dataItems = gson.fromJson(streamReader, DataItems.class);
			return dataItems.getUsers();
		}
		catch (IOException ex){
			ex.printStackTrace();
		}
		return null;
	}
	private static class DataItems {
		private List<User> users;
		List<User> getUsers() {
			return users;
		}
		void setUsers(List<User> users) {
			this.users = users;
		}
	}
}
\end{lstlisting}

Для работы с json создается объект Gson. Для сериализации данных в формат
json у этого объекта вызывается метод toJson(), в который передаются
сериализуемые данные.\par
Для упрощения работы с данными применяется вспомогательный класс
DataItems. На выходе метод toJson() возвращает строку, которая затем
сохраняется в текстовый файл.\par
Для десериализации выполняется метод fromJson(), в который передается
объект Reader с сериализованными данными и тип, к которому надо
десериализиовать данные.\par
Теперь определим основной функционал для взаимодействия с
пользователем. Изменим файл разметки следующим образом:

\begin{lstlisting}[language=XML
	, label=lst:
	]
<?xml version="1.0" encoding="utf-8"?>
<androidx.constraintlayout.widget.ConstraintLayout
    xmlns:android="http://schemas.android.com/apk/res/android"
    xmlns:app="http://schemas.android.com/apk/res-auto"
    android:layout_width="match_parent"
    android:layout_height="match_parent">
    <EditText
        android:id="@+id/nameText"
        android:layout_width="0dp"
        android:layout_height="wrap_content"
        android:hint="Введите имя"
        app:layout_constraintLeft_toLeftOf="parent"
        app:layout_constraintRight_toRightOf="parent"
        app:layout_constraintBottom_toTopOf="@id/ageText"
        app:layout_constraintTop_toTopOf="parent" />
    <EditText
        android:id="@+id/ageText"
        android:layout_width="0dp"
        android:layout_height="wrap_content"
        android:hint="Введите возраст"
        app:layout_constraintLeft_toLeftOf="parent"
        app:layout_constraintRight_toRightOf="parent"
        app:layout_constraintBottom_toTopOf="@id/addButton"
        app:layout_constraintTop_toBottomOf="@id/nameText" />
    <Button
        android:id="@+id/addButton"
        android:layout_width="0dp"
        android:layout_height="wrap_content"
        android:text="Добавить"
        android:onClick="addUser"
        app:layout_constraintLeft_toLeftOf="parent"
        app:layout_constraintRight_toRightOf="parent"
        app:layout_constraintBottom_toTopOf="@id/saveButton"
        app:layout_constraintTop_toBottomOf="@id/ageText" />
    <Button
        android:id="@+id/saveButton"
        android:layout_width="0dp"
        android:layout_height="wrap_content"
        android:text="Сохранить"
        android:onClick="save"
        app:layout_constraintLeft_toLeftOf="parent"
        app:layout_constraintRight_toLeftOf="@id/openButton"
        app:layout_constraintBottom_toTopOf="@id/list"
        app:layout_constraintTop_toBottomOf="@id/addButton"/>
    <Button
        android:id="@+id/openButton"
        android:layout_width="0dp"
        android:layout_height="wrap_content"
        android:text="Открыть"
        android:onClick="open"
        app:layout_constraintLeft_toRightOf="@id/saveButton"
        app:layout_constraintRight_toRightOf="parent"
        app:layout_constraintBottom_toTopOf="@id/list"
        app:layout_constraintTop_toBottomOf="@id/addButton"/>
    <ListView
        android:id="@+id/list"
        android:layout_width="0dp"
        android:layout_height="0dp"
        app:layout_constraintLeft_toLeftOf="parent"
        app:layout_constraintRight_toRightOf="parent"
        app:layout_constraintBottom_toBottomOf="parent"
        app:layout_constraintTop_toBottomOf="@id/openButton" />
</androidx.constraintlayout.widget.ConstraintLayout>
\end{lstlisting}

Здесь определены два текстовых поля для ввода названия модели и цены
объекта User и одна кнопка для добавления данных в список. Еще одна
кнопка выполняет сериализацию данных из списка в файл, а третья кнопка -
восстановление данных из файла.\par
Для вывода сами данных определен элемент ListView.\par
И изменим класс:

\begin{lstlisting}[language=Java
	, label=lst:
	]
private ArrayAdapter<User> adapter;
private EditText nameText, ageText;
private List<User> users;
ListView listView;

@Override
protected void onCreate(Bundle savedInstanceState) {
	super.onCreate(savedInstanceState);
	setContentView(R.layout.activity_json);
	nameText = findViewById(R.id.nameText);
	ageText = findViewById(R.id.ageText);
	listView = findViewById(R.id.list);
	users = new ArrayList<>();
	adapter = new ArrayAdapter<>(this, android.R.layout.simple_list_item_1,
			users);
	listView.setAdapter(adapter);
}
public void addUser(View view){
	String name = nameText.getText().toString();
	int age = Integer.parseInt(ageText.getText().toString());
	User user = new User(name, age);
	users.add(user);
	adapter.notifyDataSetChanged();
}
public void save(View view){
	boolean result = JSONHelper.exportToJSON(this, users);
	if(result){
		Toast.makeText(this, "Данные сохранены",
				Toast.LENGTH_LONG).show();
	}
	else{
		Toast.makeText(this, "Не удалось сохранить данные",
				Toast.LENGTH_LONG).show();
	}
}
public void open(View view){
	users = JSONHelper.importFromJSON(this);
	if(users!=null){
		adapter = new ArrayAdapter<>(this, android.R.layout.simple_list_item_1,
				users);
		listView.setAdapter(adapter);
		Toast.makeText(this, "Данные восстановлены",
				Toast.LENGTH_LONG).show();
	}
	else{
		Toast.makeText(this, "Не удалось открыть данные",
				Toast.LENGTH_LONG).show();
	}
}
\end{lstlisting}

Все данные находятся в списке users, который представляет объект
List<User>. Через адаптер этот список связывается с ListView.\par
Для сохранения и восстановления данных вызываются ранее определенные
методы в классе JSONHelper. Кнопка добавления добавляет данные в список
user, и они сразу же отображаются в ListView. При нажатии на кнопку
сохранения данные из списка users сохраняются в локальный файл. Затем с
помощью кнопки открытия мы сможем открыть ранее сохраненный файл.

\section{Работа с XML}
Одним из распространенных форматов хранения и передачи данных
является xml. Рассмотрим, как с ним работать в приложении на Android.\par
Приложение может получать данные в формате xml различными способами ---
из ресурсов, из сети и т.д. В данном случае рассмотрим ситуацию, когда файл
xml хранится в ресурсах.\par
Возьмем стандартный проект Android по умолчанию и в папке res создадим
каталог xml.\par
В этот каталог добавим новый файл, который назовем users.xml и который
будет иметь следующее содержимое:

\begin{lstlisting}[language=XML
	, label=lst:
	]
<?xml version="1.0" encoding="utf-8"?>
<users>
    <user>
        <name>Tom</name>
        <age>36</age>
    </user>
    <user>
        <name>Alice</name>
        <age>32</age>
    </user>
    <user>
        <name>Bob</name>
        <age>28</age>
    </user>
</users>
\end{lstlisting}

Это обычный файл xml, который хранит набор элементов user. Каждый
элемент характеризуется наличием двух под элементов - name и age. Условно
говоря, каждый элемент описывает пользователя, у которого есть имя и
возраст.\par
И в ту же папку добавим новый класс UserResourceParser.
Определим для класса UserResourceParser следующий код:

\begin{lstlisting}[language=Java
	, label=lst:
	]
private ArrayList<User> users;

public UserResourceParser(){
	users = new ArrayList<>();
}

public ArrayList<User> getUsers(){
	return users;
}

public boolean parse(XmlPullParser xpp){
	boolean status = true;
	User currentUser = null;
	boolean inEntry = false;
	String textValue = "";
	try{
		int eventType = xpp.getEventType();
		while(eventType != XmlPullParser.END_DOCUMENT){
			String tagName = xpp.getName();
			switch (eventType){
				case XmlPullParser.START_TAG:
					if("user".equalsIgnoreCase(tagName)){
						inEntry = true;
						currentUser = new User();
					}
					break;
				case XmlPullParser.TEXT:
					textValue = xpp.getText();
					break;
				case XmlPullParser.END_TAG:
					if(inEntry){
						if("user".equalsIgnoreCase(tagName)){
							users.add(currentUser);
							inEntry = false;
						} else if("name".equalsIgnoreCase(tagName)){
							currentUser.setName(textValue);
						} else if("age".equalsIgnoreCase(tagName)){
							currentUser.setAge(Integer.parseInt(textValue));
						}
					}
					break;
				default:
			}
			eventType = xpp.next();
		}
	}
	catch (Exception e){
		status = false;
		e.printStackTrace();
	}
	return status;
}
\end{lstlisting}

Данный класс выполняет функции парсинга xml. Распарсенные данные будут
храниться в переменной users. Непосредственно сам парсинг осуществляется
с помощью функции parse. Основную работу выполняет передаваемый в
качестве параметра объект XmlPullParser. Этот класс позволяет пробежаться
по всему документу xml и получить его содержимое.\par
Теперь изменим класс, который будет загружать ресурс xml:

\begin{lstlisting}[language=Java
	, label=lst:
	]
@Override
protected void onCreate(Bundle savedInstanceState) {
	super.onCreate(savedInstanceState);
	setContentView(R.layout.activity_xmlactivity);

	XmlPullParser xpp = getResources().getXml(R.xml.users);
	UserResourceParser parser = new UserResourceParser();
	if(parser.parse(xpp)) {
		for(User prod: parser.getUsers()){
			Log.d("XML", prod.toString());
		}
	}
}
\end{lstlisting}

\clearpage

\section*{\LARGE{Вывод}}
\addcontentsline{toc}{section}{Вывод}
В данной практической работе были получены знания о работе с контактами.
Также научились создавать провайдеры контента. В том чилсе:
определять контракт, создавать провайдер и применять их. Еще познакомились
с асинхронной загрузкой данных.\par
В дополнение к этому освоили работу с json и xml.


