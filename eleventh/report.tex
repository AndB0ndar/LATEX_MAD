\graphicspath{{./eleventh/img}}

\section*{\LARGE{Введение}}
\addcontentsline{toc}{section}{Введение}

\textbf{Задание}:
\begin{itemize}
	\item Реализовать пример с использованием потоков.
	\item Реализовать потоки, фрагменты и ViewModel.
	\item Изучить порядок и реализовать применение класса AsyncTask.
	\item Реализовать применение AsyncTask с фрагментом.
	\item Изучить работу с сетью и классом WebView. Реализовать пример.
	\item Реализовать пример загрузки данных c применением класса
		HttpsURLConnection.
\end{itemize}

\clearpage

\section*{\LARGE{Выполнение практической работы}}
\addcontentsline{toc}{section}{Выполнение практической работы}

\section{Создание потоков и визуальный интерфейс}
Когда мы запускаем приложение на Android, система создает поток, который
называется основным потоком приложения или UI-поток. Этот поток
обрабатывает все изменения и события пользовательского интерфейса.
Однако для вспомогательных операций, таких как отправка или загрузка
файла, продолжительные вычисления и т.д., мы можем создавать
дополнительные потоки.\par
Для создания новых потоков нам доcтупен стандартный функционал класса
Thread из базовой библиотеки Java из пакета java.util.concurrent, которые
особой трудности не представляют. Тем не менее трудности могут
возникнуть при обновлении визуального интерфейса из потока.\par
Например, создадим простейшее приложение с использованием потоков.
Определим разметку интерфейса в файле ресурсов, как показано
в листинге~\ref{lst:xml:thread}.

\begin{lstlisting}[language=XML
	, label=lst:xml:thread
	]
<?xml version="1.0" encoding="utf-8"?>
<androidx.constraintlayout.widget.ConstraintLayout
    xmlns:android="http://schemas.android.com/apk/res/android"
    xmlns:app="http://schemas.android.com/apk/res-auto"
    android:layout_width="match_parent"
    android:layout_height="match_parent">
    <TextView
        android:id="@+id/textView"
        android:layout_width="wrap_content"
        android:layout_height="wrap_content"
        android:text="Hello World!"
        android:textSize="22sp"
        app:layout_constraintBottom_toTopOf="@id/button"
        app:layout_constraintLeft_toLeftOf="parent"
        app:layout_constraintTop_toTopOf="parent" />
    <Button
        android:id="@+id/button"
        android:layout_width="wrap_content"
        android:layout_height="wrap_content"
        android:text="Start thread"
        app:layout_constraintTop_toBottomOf="@id/textView"
        app:layout_constraintLeft_toLeftOf="parent" />
</androidx.constraintlayout.widget.ConstraintLayout>
\end{lstlisting}

Здесь определена кнопка для запуска фонового потока, а также текстовое
поле для отображения некоторых данных, которые будут генерироваться
в запущенном потоке.\par
Далее определим в java-классе код, показанный
в листинге~\ref{lst:java:thread}.

\begin{lstlisting}[language=Java
	, label=lst:java:thread
	]
TextView textView = findViewById(R.id.textView);
Button button = findViewById(R.id.button);
button.setOnClickListener((v) -> {
		Runnable runnable = new Runnable() {
			@Override
			public void run() {
				Calendar c = Calendar.getInstance();
				int hours = c.get(Calendar.HOUR_OF_DAY);
				int minutes = c.get(Calendar.MINUTE);
				int seconds = c.get(Calendar.SECOND);
				String time = hours + ":" + minutes + ":" + seconds;
				textView.post(new Runnable() {
					public void run() {
						textView.setText(time);
					}
				});
			}
		};
		Thread thread = new Thread(runnable);
		thread.start();
});
\end{lstlisting}

Итак, здесь к кнопке прикреплен обработчик нажатия, который запускает
новый поток. Создавать и запускать поток в Java можно различными способами.
В данном случае сами действия, которые выполняются в потоке, определяются
в методе run() объекта Runnable.

\section{Потоки, фрагменты и ViewModel}
При использовании вторичных потоков следует учитывать следующий момент.
Более оптимальным способом является работа потоков с фрагментом,
нежели непосредственно с activity. Например, определим в файле
разметки, как показано в листинге~\ref{lst:xml:thread-fragments-viewmodel}.

\begin{lstlisting}[language=XML
	, label=lst:xml:thread-fragments-viewmodel
	]
<Button
	android:id="@+id/progressBtn"
	android:text="Start"
	android:layout_width="wrap_content"
	android:layout_height="wrap_content"
	app:layout_constraintBottom_toTopOf="@id/statusView"
	app:layout_constraintLeft_toLeftOf="parent"
	app:layout_constraintTop_toTopOf="parent"/>
<TextView
	android:id="@+id/statusView"
	android:text="Status"
	android:layout_width="wrap_content"
	android:layout_height="wrap_content"
	app:layout_constraintBottom_toTopOf="@id/indicator"
	app:layout_constraintLeft_toLeftOf="parent"
	app:layout_constraintTop_toBottomOf="@id/progressBtn" />
<ProgressBar
	android:id="@+id/indicator"
	style="@android:style/Widget.ProgressBar.Horizontal"
	android:layout_width="0dp"
	android:layout_height="wrap_content"
	android:max="100"
	android:progress="0"
	app:layout_constraintLeft_toLeftOf="parent"
	app:layout_constraintRight_toRightOf="parent"
	app:layout_constraintTop_toBottomOf="@id/statusView"/>
\end{lstlisting}

Здесь определена кнопка для запуска вторичной задачи и элементы TextView
и ProgressBar, которые отображают индикацию выполнения задачи.\par
В классе Activity определим код, как показано
в листинге~\ref{lst:java:thread-fragments-viewmodel}.

\begin{lstlisting}[language=Java
	, label=lst:java:thread-fragments-viewmodel
	]
btnFetch.setOnClickListener((v) -> {
	Runnable runnable = new Runnable() {
		@Override
		public void run() {
			for(; currentValue <= 100; currentValue++){
				try {
					statusView.post(new Runnable() {
						public void run() {
							indicatorBar.setProgress(currentValue);
							statusView.setText("Status: " + currentValue);
						}
					});
					Thread.sleep(400);
				} catch (InterruptedException e) {
					e.printStackTrace();
				}
			}
		}
	};
	Thread thread = new Thread(runnable);
	thread.start();
});
\end{lstlisting}

Здесь по нажатию кнопки мы запускаем задачу Runnable, в которой в цикле
от 0 до 100 изменяем показатели ProgressBar и TextView, имитируя некоторую
долгую работу.\par
Однако если в процессе работы задачи мы изменим ориентацию мобильного
устройства, то произойдет пересоздание activity, и приложение перестанет
работать должным образом.

\subsection{Добавление ViewModel}
Для подобных случаев в качестве решения проблемы предлагается использовать
ViewModel. Итак, добавим в тот же каталог новый класс MyViewModel
с кодом, показанным в листинге~\ref{lst:java:viewmodel}

\begin{lstlisting}[language=Java
	, label=lst:java:viewmodel
	]
public class ThreadFragmentViewModel extends ViewModel {
	private MutableLiveData<Boolean> isStarted = new MutableLiveData<>(false);
	private MutableLiveData<Integer> value;

	public ThreadFragmentViewModel() {
	}

	public ThreadFragmentViewModel(@NonNull Closeable... closeables) {
		super(closeables);
	}

	public LiveData<Integer> getValue() {
		if (value == null) {
			value = new MutableLiveData<Integer>(0);
		}
		return value;
	}

	public void execute(){
		if(!isStarted.getValue()){
			isStarted.postValue(true);
			Runnable runnable = new Runnable() {
				@Override
				public void run() {
					for(int i = value.getValue(); i <= 100; i++){
						try {
							value.postValue(i);
							Thread.sleep(400);
						} catch (InterruptedException e) {
							e.printStackTrace();
						}
					}
				}
			};
			Thread thread = new Thread(runnable);
			thread.start();
		}
	}
}
\end{lstlisting}

Итак, здесь определен класс MyViewModel, который унаследован от класса
ViewModel, специально предназначенного для хранения и управления состоянием
или моделью.\par
В качестве состояния здесь определены для объекта.
В первую очередь, это числовое значение, к которым будут привязан виджеты
из активити. И во-вторых, нам нужен некоторый индикатор того,
что поток уже запущен, чтобы по нажатию на кнопку не было запущено лишних
потоков.\par
Теперь задействуем наш класс MyViewModel и для этого изменим код класса
активити, как показано в листинге~\ref{}.

\begin{lstlisting}[language=Java
	, label=lst:java:
	]
ProgressBar indicatorBar = findViewById(R.id.indicator);
TextView statusView = findViewById(R.id.statusView);
Button btnFetch = findViewById(R.id.progressBtn);

ThreadFragmentViewModel model = new ViewModelProvider(this)
		.get(ThreadFragmentViewModel.class);
model.getValue().observe(this, value -> {
	indicatorBar.setProgress(value);
	statusView.setText("Status: " + value);
});
btnFetch.setOnClickListener(v -> model.execute());
\end{lstlisting}

Метод observe() в качестве первого параметра принимает владельца функции
обсервера --- в данном случае текущий объект MainActivity.
А в качестве второго параметра --- функцию обсервера
(а точнее объект интерфейса Observer). Функция обсервера принимает один
параметр --- новое значение отслеживаемой переменной (то есть в данном
случае переменной value). Получив новое значение переменной value,
мы изменяем параметры виджетов.\par
Таким образом, при каждом изменении значения в переменной value
виджеты получат ее новое значение.

\subsection{Использование фрагментов}
Аналогично мы можем использовать фрагементы. Итак, добавим в проект
новый фрагмент, который назовем ProgressFragment.\par
Определим для него новый файл разметки интерфейса
fragment\_progress.xml, как показано
в листинге~\ref{lst:xml:progress-fragment}.

\begin{lstlisting}[language=XML
	, label=lst:xml:progress-fragment
	]
<Button
	android:id="@+id/progressBtn"
	android:text="Start"
	android:layout_width="wrap_content"
	android:layout_height="wrap_content"
	app:layout_constraintBottom_toTopOf="@id/statusView"
	app:layout_constraintLeft_toLeftOf="parent"
	app:layout_constraintTop_toTopOf="parent"/>
<TextView
	android:id="@+id/statusView"
	android:text="Status"
	android:layout_width="wrap_content"
	android:layout_height="wrap_content"
	app:layout_constraintBottom_toTopOf="@id/indicator"
	app:layout_constraintLeft_toLeftOf="parent"
	app:layout_constraintTop_toBottomOf="@id/progressBtn" />
<ProgressBar
	android:id="@+id/indicator"
	style="@android:style/Widget.ProgressBar.Horizontal"
	android:layout_width="0dp"
	android:layout_height="wrap_content"
	android:max="100"
	android:progress="0"
	app:layout_constraintLeft_toLeftOf="parent"
	app:layout_constraintRight_toRightOf="parent"
	app:layout_constraintTop_toBottomOf="@id/statusView"/>
\end{lstlisting}

Сам класс фрагмента ProgressFragment изменим следующим образом,
как показано в листинге~\ref{lst:java:progress-fragment}.

\begin{lstlisting}[language=Java
	, label=lst:java:progress-fragment
	]
@Override
public View onCreateView(LayoutInflater inflater, ViewGroup container,
						 Bundle savedInstanceState) {
	View view = inflater.inflate(R.layout.fragment_progress, container, false);

	ProgressBar indicatorBar = (ProgressBar) view.findViewById(R.id.indicator);
	TextView statusView = (TextView) view.findViewById(R.id.statusView);
	Button btnFetch = (Button) view.findViewById(R.id.progressBtn);

	ThreadFragmentViewModel model = new ViewModelProvider(requireActivity())
		.get(ThreadFragmentViewModel.class);

	model.getValue().observe(getViewLifecycleOwner(), value -> {
		indicatorBar.setProgress(value);
		statusView.setText("Status: " + value);
	});
	btnFetch.setOnClickListener(v -> model.execute());

	return view;
}
\end{lstlisting}

Здесь аналогичным образом применяется класс MyViewModel. Единственно
для получения ассоциируемой с фрагментом Activity здесь применяется
метод requireActivity(). А для получения владельца жизненного цикла -
метод getViewLifecycleOwner.\par
Теперь свяжем фрагмент с activity. Для этого определим в файле
ресурсов код, как показано в листинге~\ref{lst:xml:main:progress-fragment}.

\begin{lstlisting}[language=XML
	, label=lst:xml:main:progress-fragment
	]
<?xml version="1.0" encoding="utf-8"?>
<androidx.fragment.app.FragmentContainerView
    xmlns:android="http://schemas.android.com/apk/res/android"
    android:id="@+id/fragment_container_view"
    android:layout_width="match_parent"
    android:layout_height="match_parent"
    android:name="com.example.eleven.ProgressFragment" />
\end{lstlisting}

\section{Класс AsyncTask}
В прошлых материалах был описан общий подход, который применяется
сейчас для запуска в приложении нового потока и обновления в нем
пользовательского интерфейса. Рассмотрим другой подход, который
представляет класс AsyncTask. Хотя примение AsyncTask в современных
приложениях Android устарел. Тем не менее, поскольку он по-прежнему
широко применяется, также рассмотрим его.\par
Итак, создадим простейшее приложение с использованием AsyncTask.
Определим следующую разметку интерфейса в файле ресурсов,
как показано в листинге~\ref{lst:xml:async-task}.

\begin{lstlisting}[language=XML
	, label=lst:xml:async-task
	]
<LinearLayout
	android:layout_width="match_parent"
	android:layout_height="wrap_content">
	<TextView
		android:layout_width="wrap_content"
		android:layout_height="wrap_content"
		android:textSize="22sp"
		android:id="@+id/clicksView"
		android:text="Clicks: 0"/>
	<Button
		android:layout_width="wrap_content"
		android:layout_height="wrap_content"
		android:id="@+id/clicksBtn"
		android:text="Click" />
</LinearLayout>

<Button
	android:id="@+id/progressBtn"
	android:text="Start"
	android:layout_width="wrap_content"
	android:layout_height="wrap_content" />

<TextView
	android:id="@+id/statusView"
	android:text="Status"
	android:layout_width="wrap_content"
	android:layout_height="wrap_content" />
<ProgressBar
	android:id="@+id/indicator"
	style="@android:style/Widget.ProgressBar.Horizontal"
	android:layout_width="match_parent"
	android:layout_height="wrap_content"
	android:max="100"
	android:progress="0" />
\end{lstlisting}

Здесь определена кнопка для запуска фонового потока, а также текстовое
поле и прогрессбар для индикации выполнения задачи. Кроме того, здесь
определены дополнительная кнопка, которая увеличивает числов кликов,
и текстовое поле, оторое выводит число кликов.\par
Далее определим в классе активити код, как показано
в листинге~\ref{lst:java:async-task}.

\begin{lstlisting}[language=Java
	, label=lst:java:async-task
	]
int[] integers=null;
int clicks = 0;
ProgressBar indicatorBar;
TextView statusView;
TextView clicksView;
Button progressBtn;
Button clicksBtn;

@Override
protected void onCreate(Bundle savedInstanceState) {
	super.onCreate(savedInstanceState);
	setContentView(R.layout.activity_async_task);

	integers = new int[100];
	for (int i = 0; i < 100; i++) {
		integers[i] = i + 1;
	}

	indicatorBar = (ProgressBar) findViewById(R.id.indicator);
	statusView = findViewById(R.id.statusView);
	progressBtn = findViewById(R.id.progressBtn);

	progressBtn.setOnClickListener((v) -> new ProgressTask().execute());

	clicksView = findViewById(R.id.clicksView);
	clicksBtn = findViewById(R.id.clicksBtn);

	clicksBtn.setOnClickListener((v) -> {
		clicks++;
		clicksView.setText("Clicks: " + clicks);
	});
}

class ProgressTask extends AsyncTask<Void, Integer, Void> {
	@Override
	protected Void doInBackground(Void... unused) {
		for (int i = 0; i < integers.length; i++) {
			publishProgress(i);
			SystemClock.sleep(400);
		}
		return(null);
	}

	@Override
	protected void onProgressUpdate(Integer... items) {
		indicatorBar.setProgress(items[0] + 1);
		statusView.setText("Status: " + String.valueOf(items[0] + 1));
	}

	@Override
	protected void onPostExecute(Void unused) {
		Toast.makeText(getApplicationContext()
				, "Work complete"
				, Toast.LENGTH_SHORT).show();
	}
}
\end{lstlisting}

AsyncTask содержит четыре метода, которые можно переопределить:

\begin{itemize}
	\item \texttt{doInBackground()}: выполняется в фоновом потоке,
		должен возвращать определенный результат;
	\item \texttt{onPreExecute()}: вызывается из главного потока перед
		запуском метода doInBackground();
	\item \texttt{onPostExecute()}: выполняется из главного потока после
		завершения работы метода doInBackground();
	\item \texttt{onProgressUpdate()}: позволяет сигнализировать
		пользователю о выполнении фонового потока.
\end{itemize}

\subsection{AsyncTask и фрагменты}
При использовании AsyncTask следует учитывать следующий момент. Более
оптимальным способом является работа AsyncTask с фрагментом, нежели
непосредственно с activity. Например, если мы возьмем проект из прошлого
материала, запустим приложение и изменим ориентацию мобильного
устройства, то произойдет пересоздание activity. В случае изменения
орентации устройства поток AsyncTask будет продолжать обращаться к
старой activity, вместо новой. Поэтому в этом случае лучше использовать
фрагменты.\par
Определим новый файл разметки интерфейса
fragment\_progress.xml, как показано
в листинге~\ref{lst:xml:async-task:fragment}.

\begin{lstlisting}[language=XML
	, label=lst:xml:async-task:fragment
	]
<?xml version="1.0" encoding="utf-8"?>
<LinearLayout xmlns:android="http://schemas.android.com/apk/res/android"
    android:id="@+id/activity_main"
    android:orientation="vertical"
    android:layout_width="match_parent"
    android:layout_height="match_parent">
    <fragment
        android:id="@+id/progressFragment"
        android:layout_width="match_parent"
        android:layout_height="match_parent"
        android:name="com.example.eleven.ProgressSecondFragment"/>
</LinearLayout>
\end{lstlisting}

Сам класс фрагмента ProgressFragment заполним следующим образом кодом,
показанным в листинге~\ref{lst:java:async-task:fragment}.

\begin{lstlisting}[language=Java
	, label=lst:java:async-task:fragment
	]
int[] integers=null;
ProgressBar indicatorBar;
TextView statusView;
@Override
public void onCreate(Bundle savedInstanceState) {
	super.onCreate(savedInstanceState);
	setRetainInstance(true);
}
@Override
public View onCreateView(LayoutInflater inflater, ViewGroup container,
						 Bundle savedInstanceState) {
	View view = inflater.inflate(R.layout.fragment_progress, container, false);
	integers = new int[100];
	for(int i=0;i<100;i++) {
		integers[i] = i + 1;
	}
	indicatorBar = view.findViewById(R.id.indicator);
	statusView = view.findViewById(R.id.statusView);
	Button btnFetch = view.findViewById(R.id.progressBtn);
	btnFetch.setOnClickListener((v) -> {
			new ProgressTask().execute();
	});
	return view;
}
class ProgressTask extends AsyncTask<Void, Integer, Void> {
	@Override
	protected Void doInBackground(Void... unused) {
		for (int i = 0; i<integers.length;i++) {
			publishProgress(i);
			SystemClock.sleep(400);
		}
		return null;
	}
	@Override
	protected void onProgressUpdate(Integer... items) {
		indicatorBar.setProgress(items[0]+1);
		statusView.setText("Status: " + String.valueOf(items[0]+1));
	}
	@Override
	protected void onPostExecute(Void unused) {
		Toast.makeText(getActivity(), "Work compele",
						Toast.LENGTH_SHORT)
				.show();
	}
}
\end{lstlisting}

\section{Работа с сетью. WebView}
\subsection{WebView}
WebView представляет простейший элемент для рендеринга html-кода,
базирующийся на движке WebKit. Благодаря этому мы можем использовать
WebView как примитивный веб-браузер, просматривая через него контент из
сети интернет. Использование движка WebKit гарантирует, что отображение
контента будет происходить примерно такжe, как и в других браузерах,
построенных на этом движке --- Google Chrome и Safari.\par
Работать с WebView очень просто. Определим данный элемент в разметке
layout, как показано в листинге~\ref{lst:xml:webview}.

\begin{lstlisting}[language=XML
	, label=lst:xml:webview
	]
<?xml version="1.0" encoding="utf-8"?>
<androidx.constraintlayout.widget.ConstraintLayout
    xmlns:android="http://schemas.android.com/apk/res/android"
    xmlns:app="http://schemas.android.com/apk/res-auto"
    android:layout_width="match_parent"
    android:layout_height="match_parent">
    <WebView
        android:id="@+id/webBrowser"
        android:layout_width="0dp"
        android:layout_height="0dp"
        app:layout_constraintLeft_toLeftOf="parent"
        app:layout_constraintRight_toRightOf="parent"
        app:layout_constraintTop_toTopOf="parent"
        app:layout_constraintBottom_toBottomOf="parent"/>
</androidx.constraintlayout.widget.ConstraintLayout>
\end{lstlisting}

Для получения доступа к интернету из приложения, необходимо указать в
файле манифеста AndroidManifest.xml соответствующее разрешение:

\begin{verbatim}
<uses-permission android:name="android.permission.INTERNET"/>
\end{verbatim}

Чтобы загрузить определенную страницу в WebView, через метод loadUrl()
надо установить ее адрес (Рисунок~\ref{}).

\begin{lstlisting}[language=Java
	, label=lst:
	]
@Override
protected void onCreate(Bundle savedInstanceState) {
	super.onCreate(savedInstanceState);
	setContentView(R.layout.activity_web_view);

	WebView browser = findViewById(R.id.webBrowser);
	browser.loadUrl("https://www.mirea.ru/");
}
\end{lstlisting}

\subsection{JavaScript}

По умолчанию в WebView отключен javascript, чтобы его включить надо
применить метод setJavaScriptEnabled(true) объекта WebSettings:

\begin{verbatim}
import android.webkit.WebSettings;
//.....................................
WebView browser = findViewById(R.id.webBrowser);
WebSettings webSettings = browser.getSettings();
webSettings.setJavaScriptEnabled(true);
\end{verbatim}

\subsection{Load данных и класс HttpURLConnection}
На сегодняшний день если не все, то большинство Android-устройств имеют
доступ к сети интернет. А большое количество мобильных приложений так
или иначе взаимодействуют с средой интернет: загружают файлы,
авторизуются и получают информацию с внешних веб-сервисов и т.д.
Рассмотрим, как мы можем использовать в своем приложении доступ к сети
интернет.\par
Среди стандартных элементов нам доступен виджет WebView, который
может загружать контент с определенного url-адреса. Но этим возможности
работы с сетью в Android не ограничиваются. Для получения данных с
определенного интернет-ресурса мы можем использовать классы
HttpUrlConnection (для протокола HTTP) и HttpsUrlConnection (для
протокола HTTPS) из стандартной библиотеки Java.\par
Итак, создадим новый проект с пустой MainActivity. Первым делом для
работы с сетью нам надо установить в файле манифеста
AndroidManifest.xml соответствующее разрешение:

\begin{verbatim}
<uses-permission android:name="android.permission.INTERNET"/>
\end{verbatim}

В файле разметки определим код, как показано
в листинге~\ref{lst:xml:http-url-connection}.

\begin{lstlisting}[language=XML
	, label=lst:xml:http-url-connection
	]
<?xml version="1.0" encoding="utf-8"?>
<androidx.constraintlayout.widget.ConstraintLayout
    xmlns:android="http://schemas.android.com/apk/res/android"
    xmlns:app="http://schemas.android.com/apk/res-auto"
    android:layout_width="match_parent"
    android:layout_height="match_parent" >
    <Button
        android:id="@+id/downloadBtn"
        android:layout_width="0dp"
        android:layout_height="wrap_content"
        android:text="Load"
        app:layout_constraintLeft_toLeftOf="parent"
        app:layout_constraintRight_toRightOf="parent"
        app:layout_constraintTop_toTopOf="parent" />
    <WebView
        android:id="@+id/webView"
        android:layout_width="0dp"
        android:layout_height="0dp"
        app:layout_constraintLeft_toLeftOf="parent"
        app:layout_constraintRight_toRightOf="parent"
        app:layout_constraintTop_toBottomOf="@id/downloadBtn"
        app:layout_constraintBottom_toTopOf="@id/scrollView" />
    <ScrollView
        android:id="@+id/scrollView"
        android:layout_width="0dp"
        android:layout_height="0dp"
        app:layout_constraintLeft_toLeftOf="parent"
        app:layout_constraintRight_toRightOf="parent"
        app:layout_constraintTop_toBottomOf="@id/webView"
        app:layout_constraintBottom_toBottomOf="parent">
        <TextView android:id="@+id/content"
            android:layout_width="match_parent"
            android:layout_height="wrap_content" />
    </ScrollView>
</androidx.constraintlayout.widget.ConstraintLayout>
\end{lstlisting}

Здесь определена кнопка для загрузки данных, а сами данные для примера
загружаются одновременно в виде строки в текстовое поле и в элемент
WebView. Так как данных может быть очень много, то текстовое поле
помещено в элемент ScrollView.\par
Поскольку загрузка данных может занять некоторое время, то обращение к
интернет-ресурсу определим в отдельном потоке и для этого напишем код,
как показано в листинге~\ref{lst:java:http-url-connection}.

\begin{lstlisting}[language=Java
	, label=lst:java:http-url-connection
	]
private static String url = "https://github.com/AndB0ndar";

@Override
protected void onCreate(Bundle savedInstanceState) {
	super.onCreate(savedInstanceState);
	setContentView(R.layout.activity_http_urlconnection);

	TextView contentView = findViewById(R.id.content);

	WebView webView = findViewById(R.id.webView);
	webView.getSettings().setJavaScriptEnabled(true);
	Button btnFetch = findViewById(R.id.downloadBtn);
	btnFetch.setOnClickListener((v) -> {
		contentView.setText("Load...");
		new Thread(new Runnable() {
			public void run() {
				try{
					String content = getContent(url);
					webView.post(new Runnable() {
						public void run() {
							webView.loadDataWithBaseURL(url
								, content,"text/html"
								, "UTF-8", url);
							Toast.makeText(getApplicationContext()
								, "Data load"
								, Toast.LENGTH_SHORT)
								.show();
						}
					});
					contentView.post(new Runnable() {
						public void run() {
							contentView.setText(content);
						}
					});
				}
				catch (IOException ex){
					contentView.post(new Runnable() {
						public void run() {
							contentView.setText("Error: "
								+ ex.getMessage());
							Toast.makeText(getApplicationContext()
								, "Error"
								, Toast.LENGTH_SHORT)
								.show();
						}
					});
				}
			}
		}).start();
	});
}

private String getContent(String path) throws IOException {
	BufferedReader reader=null;
	InputStream stream = null;
	HttpsURLConnection connection = null;
	try {
		URL url=new URL(path);
		connection =(HttpsURLConnection)url.openConnection();
		connection.setRequestMethod("GET");
		connection.setReadTimeout(10000);
		connection.connect();
		stream = connection.getInputStream();
		reader= new BufferedReader(new InputStreamReader(stream));
		StringBuilder buf=new StringBuilder();
		String line;
		while ((line=reader.readLine()) != null) {
			buf.append(line).append("\n");
		}
		return(buf.toString());
	}
	finally {
		if (reader != null) {
			reader.close();
		}
		if (stream != null) {
			stream.close();
		}
		if (connection != null) {
			connection.disconnect();
		}
	}
}
\end{lstlisting}

\clearpage

\section*{\LARGE{Вывод}}
\addcontentsline{toc}{section}{Вывод}
В данной практической работе были получены знания об создании и управлении
потоками. Научились работать как с объектами Thread, так и с AsyncTask.
Узнали как работать с ViewModel и как применять их со фрагментами.\par
Также узнали как работать с сетью. Познакомились с виджетом WebView,
и научились с помощью него открывать страницы из интернета.

