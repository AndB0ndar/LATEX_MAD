\graphicspath{{./twelfth/img}}

\section*{\LARGE{Введение}}
\addcontentsline{toc}{section}{Введение}

\textbf{Задание}:
\begin{itemize}
	\item Реализовать приложение для работы с видео материалами в
		стандартном наборе виджетов применяя класс VideoView
	\item С помощью класса MediaController добавить к VideoView
		дополнительно элементы управления.
	\item Реализовать воспроизведение файла из интернета.
	\item Реализовать приложение для работы с аудио материалами, применяя
		класс MediaPlayer
	\item Реализовать приложение для работы с анимацией, применяя
		технологию Cell animation.
	\item Реализовать приложение для работы с анимацией, применяя
		технологию Tween-animation. Создать приложение с сервисом со
		всеми необходимыми этапами жизненного цикла, применяя класс
		Service.
	\item Реализовать диалоговые окна, применяя технологии DatePickerDialog и
		TimePickerDialog, которые позволяют выбрать дату и время.
	\item Применяя класс DialogFragment создать свои диалоговые окна.
	\item Реализовать передачу данных в диалоговое окно, как и в любой
		фрагмент, с помощью объекта Bundle.
	\item взаимодействие диалогового окна с Activity
\end{itemize}

\clearpage

\section*{\LARGE{Выполнение практической работы}}
\addcontentsline{toc}{section}{Выполнение практической работы}

\section{Работа с видео}
Для работы с видеоматериалами в стандартном наборе виджетов Android
определен класс VideoView, который позволяет воспроизводить видео.\par
Android поддерживает большинство распространенных типов видеофайлов,
в частности, 3GPP (.3gp), WebM (.webm), Matroska (.mkv), MPEG-4 (.mp4).\par
VideoView может работать как с роликами, размещенными на мобильном
устройстве, так и с видеоматериалами из сети. В данном случае используем
видеоролик, размещенный локально. Для этого добавим в проект какойнибудь видеоролик. Обычно видеоматериалы помещают в проекте в папку
res/raw. По умолчанию проект не содержит подобной папки, поэтому
добавим в каталог res подпапку raw.

Теперь определим функционал для его воспроизведения. Для этого в файле
разметки укажем следующий код:

\begin{lstlisting}[language=XML
	, label=lst:
	]
<?xml version="1.0" encoding="utf-8"?>
<androidx.constraintlayout.widget.ConstraintLayout
    xmlns:android="http://schemas.android.com/apk/res/android"
    xmlns:app="http://schemas.android.com/apk/res-auto"
    android:layout_width="match_parent"
    android:layout_height="match_parent">
    <Button
        android:id="@+id/playButton"
        android:layout_width="0dp"
        android:layout_height="wrap_content"
        android:text="Play"
        android:onClick="play"
        app:layout_constraintBottom_toTopOf="@id/videoPlayer"
        app:layout_constraintLeft_toLeftOf="parent"
        app:layout_constraintRight_toLeftOf="@id/pauseButton"
        app:layout_constraintTop_toTopOf="parent" />
        <Button
        android:id="@+id/pauseButton"
        android:layout_width="0dp"
        android:layout_height="wrap_content"
        android:text="Pause"
        android:onClick="pause"
        app:layout_constraintBottom_toTopOf="@id/videoPlayer"
        app:layout_constraintLeft_toRightOf="@id/playButton"
        app:layout_constraintRight_toLeftOf="@id/stopButton"
        app:layout_constraintTop_toTopOf="parent"/>
    <Button
        android:id="@+id/stopButton"
        android:layout_width="0dp"
        android:layout_height="wrap_content"
        android:text="Stop"
        android:onClick="stop"
        app:layout_constraintBottom_toTopOf="@id/videoPlayer"
        app:layout_constraintLeft_toRightOf="@id/pauseButton"
        app:layout_constraintRight_toRightOf="parent"
        app:layout_constraintTop_toTopOf="parent" />
    <VideoView android:id="@+id/videoPlayer"
        android:layout_height="0dp"
        android:layout_width="0dp"
        app:layout_constraintBottom_toBottomOf="parent"
        app:layout_constraintLeft_toLeftOf="parent"
        app:layout_constraintRight_toRightOf="parent"
        app:layout_constraintTop_toBottomOf="@id/playButton"/>
</androidx.constraintlayout.widget.ConstraintLayout>
\end{lstlisting}

Для управления воспроизведением видео здесь определены три кнопки: для
запуска видео, для паузы и для его остановки.
И также изменим код:

\begin{lstlisting}[language=Java
	, label=lst:
	]
VideoView videoPlayer;

@Override
protected void onCreate(Bundle savedInstanceState) {
	super.onCreate(savedInstanceState);
	setContentView(R.layout.activity_video);

	videoPlayer = findViewById(R.id.videoPlayer);
	Uri myVideoUri= Uri.parse("android.resource://"
		+ getPackageName()
		+ "/" + R.raw.i_kak_prosto);
	videoPlayer.setVideoURI(myVideoUri);
}

public void play(View view){
	videoPlayer.start();
}

public void pause(View view){
	videoPlayer.pause();
}

public void stop(View view){
	videoPlayer.stopPlayback();
	videoPlayer.resume();
}
\end{lstlisting}

\subsection{MediaController}
С помощью класса MediaController мы можем добавить к VideoView
дополнительно элементы управления. Для этого изменим код:

\begin{lstlisting}[language=Java
	, label=lst:
	]
VideoView videoPlayer;

@Override
protected void onCreate(Bundle savedInstanceState) {
	super.onCreate(savedInstanceState);
	setContentView(R.layout.activity_video);

	videoPlayer = findViewById(R.id.videoPlayer);
	Uri myVideoUri= Uri.parse( "android.resource://"
		+ getPackageName()
		+ "/" + R.raw.i_kak_prosto);
	videoPlayer.setVideoURI(myVideoUri);

	MediaController mediaController = new MediaController(this);
	videoPlayer.setMediaController(mediaController);
	mediaController.setMediaPlayer(videoPlayer);
}

public void play(View view){
	videoPlayer.start();
}

public void pause(View view){
	videoPlayer.pause();
}

public void stop(View view){
	videoPlayer.stopPlayback();
	videoPlayer.resume();
}
\end{lstlisting}

И если мы запустим приложения, то при касании по VideoView внизу
появятся инструменты для управления видео.

\subsection{Воспроизведение файла из интернета}
VideoView поддерживает воспроизведение файла из интернета. Но чтобы это
стало возможно, необходимо в файле AndroidManifest.xml установить
разрешение android.permission.INTERNET, так как мы получаем данные из
интернета:

\begin{verbatim}
<uses-permission android:name="android.permission.INTERNET" />
\end{verbatim}

Далее изменим класс:

\begin{lstlisting}[language=Java
	, label=lst:
	]
VideoView videoPlayer;

@Override
protected void onCreate(Bundle savedInstanceState) {
	super.onCreate(savedInstanceState);
	setContentView(R.layout.activity_internet_video);

	videoPlayer = findViewById(R.id.videoPlayer);
	videoPlayer.setVideoPath("https://www.youtube.com/watch?v=YnzCSdzLdx8");

	MediaController mediaController = new MediaController(this);
	videoPlayer.setMediaController(mediaController);
	mediaController.setMediaPlayer(videoPlayer);
}

public void play(View view){
	videoPlayer.start();
}

public void pause(View view){
	videoPlayer.pause();
}

public void stop(View view){
	videoPlayer.stopPlayback();
	videoPlayer.resume();
}
\end{lstlisting}

Здесь нам надо в метод \texttt{videoPlayer.setVideoPath()} передать
интернет-адрес воспроизводимого файла.

\section{Воспроизведение аудио}
Для воспроизведения музыки и других аудиоматериалов Android
предоставляет класс MediaPlayer.\par
Чтобы воспроизводить аудио, MediaPlayer должен знать, какой именно
ресурс (файл) нужно производить. Установить нужный ресурс для
воспроизведения можно тремя способами:

\begin{itemize}
	\item в метод \texttt{create()} объекта MediaPlayer передается id ресурса,
		представляющего аудиофайл
	\item в метод \texttt{create()} объекта MediaPlayer передается объект Uri,
		представляющего аудиофайл
	\item в метод \texttt{setDataSource()} объекта MediaPlayer передается
		полный путь к аудиофайлу
\end{itemize}

Для управления аудиопотоком определим в файле разметки три
кнопки:

\begin{lstlisting}[language=XML
	, label=lst:
	]
<?xml version="1.0" encoding="utf-8"?>
<androidx.constraintlayout.widget.ConstraintLayout
    xmlns:android="http://schemas.android.com/apk/res/android"
    xmlns:app="http://schemas.android.com/apk/res-auto"
    android:layout_width="match_parent"
    android:layout_height="match_parent">
    <Button
        android:id="@+id/playButton"
        android:layout_width="0dp"
        android:layout_height="wrap_content"
        android:text="Play"
        android:onClick="play"
        app:layout_constraintLeft_toLeftOf="parent"
        app:layout_constraintRight_toLeftOf="@id/pauseButton"
        app:layout_constraintTop_toTopOf="parent" />
    <Button
        android:id="@+id/pauseButton"
        android:layout_width="0dp"
        android:layout_height="wrap_content"
        android:text="Pause"
        android:onClick="pause"
        app:layout_constraintLeft_toRightOf="@id/playButton"
        app:layout_constraintRight_toLeftOf="@id/stopButton"
        app:layout_constraintTop_toTopOf="parent"/>
    <Button
        android:id="@+id/stopButton"
        android:layout_width="0dp"
        android:layout_height="wrap_content"
        android:text="Stop"
        android:onClick="stop"
        app:layout_constraintLeft_toRightOf="@id/pauseButton"
        app:layout_constraintRight_toRightOf="parent"
        app:layout_constraintTop_toTopOf="parent" />
    <SeekBar
        android:id="@+id/volumeControl"
        android:layout_width="0dp"
        android:layout_height="wrap_content"
        android:layout_marginTop="32dp"
        app:layout_constraintTop_toBottomOf="@id/playButton"
        app:layout_constraintRight_toRightOf="parent"
        app:layout_constraintLeft_toLeftOf="parent" />
</androidx.constraintlayout.widget.ConstraintLayout>
\end{lstlisting}

И изменим код класса:

\begin{lstlisting}[language=Java
	, label=lst:
	]
MediaPlayer mPlayer;
Button playButton, pauseButton, stopButton;
SeekBar volumeControl;
AudioManager audioManager;

@Override
protected void onCreate(Bundle savedInstanceState) {
	super.onCreate(savedInstanceState);
	setContentView(R.layout.activity_audio);

	mPlayer= MediaPlayer.create(this, R.raw.saver);
	mPlayer.setOnCompletionListener(new MediaPlayer.OnCompletionListener() {
		@Override
		public void onCompletion(MediaPlayer mp) {
			stopPlay();
		}
	});

	playButton = findViewById(R.id.playButton);
	pauseButton = findViewById(R.id.pauseButton);
	stopButton = findViewById(R.id.stopButton);
	audioManager = (AudioManager) getSystemService(Context.AUDIO_SERVICE);

	int maxVolume = audioManager.getStreamMaxVolume(AudioManager.STREAM_MUSIC);
	int curValue = audioManager.getStreamVolume(AudioManager.STREAM_MUSIC);

	volumeControl = findViewById(R.id.volumeControl);
	volumeControl.setMax(maxVolume);
	volumeControl.setProgress(curValue);
	volumeControl.setOnSeekBarChangeListener(
		new SeekBar.OnSeekBarChangeListener() {
		@Override
		public void onProgressChanged(SeekBar seekBar
			, int progress, boolean fromUser) {
			audioManager.setStreamVolume(AudioManager.STREAM_MUSIC
				, progress, 0);
		}
		@Override
		public void onStartTrackingTouch(SeekBar seekBar) {
		}
		@Override
		public void onStopTrackingTouch(SeekBar seekBar) {
		}
	});
	pauseButton.setEnabled(false);
	stopButton.setEnabled(false);
}

private void stopPlay(){
	mPlayer.stop();
	pauseButton.setEnabled(false);
	stopButton.setEnabled(false);
	try {
		mPlayer.prepare();
		mPlayer.seekTo(0);
		playButton.setEnabled(true);
	}
	catch (Throwable t) {
		Toast.makeText(this, t.getMessage(), Toast.LENGTH_SHORT).show();
	}
}

public void play(View view){
	mPlayer.start();
	playButton.setEnabled(false);
	pauseButton.setEnabled(true);
	stopButton.setEnabled(true);
}

public void pause(View view){
	mPlayer.pause();
	playButton.setEnabled(true);
	pauseButton.setEnabled(false);
	stopButton.setEnabled(true);
}

public void stop(View view){
	stopPlay();
}

@Override
public void onDestroy() {
	super.onDestroy();
	if (mPlayer.isPlaying()) {
		stopPlay();
	}
}
\end{lstlisting}

Обработчик каждой кнопки кроме вызова определенного метода у
MediaPlayer также переключает доступность кнопок.\par
И если запуск и приостановка воспроизведения особых сложностей не
вызывает, то при обработки полной остановки воспроизведения мы можем
столкнуться с рядом трудностей. В частности, когда мы выходим из
приложения - полностью закрываем его через диспетчер приложений либо
нажимаем на кнопку Назад, то у нас для текущей Activity вызывается метод
onDestroy, activity уничтожается, но MediaPlayer продолжает работать. Если
мы вернемся к приложению, то activity будет создана заново, но с помощью
кнопок мы не сможем управлять воспроизведением. Поэтому в данном
случае переопределяем метод onDestroy, в котором завершаем
воспроизведение.\par
Для корректного завершения также определен обработчик естественного
завершения воспроизведения OnCompletionListener, действие которого
будет аналогично нажатию на кнопку "Stop".\par
Для управления громкостью звука применяется класс AudioManager.
А с помощью вызова
\texttt{audioManager.setStreamVolume(AudioManager.STREAM\_MUSIC,
progress, 0);} в качестве второго параметра можно передать нужное значение
громкости.

\section{Анимация}
\subsection{Cell-анимация}
Cell animation или анимация фреймов представляет собой технику анимации,
при которой ряд изображений или кадров/фреймов последовательно сменяют
друг друга за короткий промежуток времени. Подобная техника довольно
распространена при создании мультфильмов.\par
При достаточно быстрой смене кадров получится динамический эффект
зайца, забрасывающего мяч в баскетбольную корзину. Теперь рассмотрим,
как сделать подобную анимацию в приложении Android.
Во-первых, нам надо добавить все эти изображения в проект в папку
res/drawable. И в эту же папку добавим новый xml-файл. Назовем его
rabit\_animation.xml и поместим в него следующее содержимое:

\begin{lstlisting}[language=XML
	, label=lst:
	]
<?xml version="1.0" encoding="utf-8"?>
<animation-list xmlns:android="http://schemas.android.com/apk/res/android"
 android:oneshot="false" >
 <item android:drawable="@drawable/a1" android:duration="250" />
 <item android:drawable="@drawable/a2" android:duration="250" />
 <item android:drawable="@drawable/a3" android:duration="250" />
 <item android:drawable="@drawable/a4" android:duration="250" />
 <item android:drawable="@drawable/a5" android:duration="250" />
 <item android:drawable="@drawable/a6" android:duration="250" />
 <item android:drawable="@drawable/a7" android:duration="250" />
</animation-list>
\end{lstlisting}

В разметке интерфейса для отображения анимации используется элемент
ImageView:

\begin{lstlisting}[language=XML
	, label=lst:
	]
<?xml version="1.0" encoding="utf-8"?>
<androidx.constraintlayout.widget.ConstraintLayout xmlns:android="http://schemas.android.com/apk/res/android"
    xmlns:app="http://schemas.android.com/apk/res-auto"
    xmlns:tools="http://schemas.android.com/tools"
    android:layout_width="match_parent"
    android:layout_height="match_parent"
    tools:context=".CellAnimationActivity">

    <ImageView android:id="@+id/animationView"
        android:layout_width="0dp"
        android:layout_height="0dp"
        android:adjustViewBounds="true"
        app:layout_constraintBottom_toBottomOf="parent"
        app:layout_constraintLeft_toLeftOf="parent"
        app:layout_constraintRight_toRightOf="parent"
        app:layout_constraintTop_toTopOf="parent"/>
</androidx.constraintlayout.widget.ConstraintLayout>
\end{lstlisting}

Далее изменим код, чтобы запустить анимацию:

\begin{lstlisting}[language=Java
	, label=lst:
	]
@Override
protected void onCreate(Bundle savedInstanceState) {
	super.onCreate(savedInstanceState);
	setContentView(R.layout.activity_cell_animation);

	ImageView img = findViewById(R.id.animationView);
	img.setBackgroundResource(R.drawable.rabit_animation);
	AnimationDrawable frameAnimation = (AnimationDrawable) img.getBackground();
	img.setOnClickListener((v) -> frameAnimation.start());
}
\end{lstlisting}

С помощью метода \texttt{setBackgroundResource()} объекта ImageView
устанавливается ресурс анимации. Затем из ImageView получаем собственно
объект анимации AnimationDrawable и по нажатию на ImageView запускаем
анимацию с помощью метода \texttt{start()}.

\subsection{Tween-анимация}
Tween-анимация представляет анимацию различных свойств объекта, при
которой система сама рассчитывает некоторые промежуточные значения с
помощью определенного алгоритма, который называется интерполяцией. В 
Android алгоритм интерполяции определяется встроенным классом
Animation.\par
От данного класса наследуются классы, которые описывают конкретные
типы анимаций, такие как AlphaAnimation (изменение прозрачности),
RotateAnimation (анимация вращения), ScaleAnimation (анимация
масштабирования), TranslateAnimation (анимация перемещения).\par
Мы можем определить анимацию как в коде java, так и в файле xml. Для
хранения ресурсов анимации в папке res предназначена папка anim. По
умолчанию данная папка отсутствует в проекте, поэтому создадим ее.\par
Определим в этом файле следующее содержание:

\begin{lstlisting}[language=XML
	, label=lst:
	]
<?xml version="1.0" encoding="utf-8"?>
<set xmlns:android="http://schemas.android.com/apk/res/android"
    android:interpolator="@android:anim/linear_interpolator">
    <scale
        android:fromXScale="0.3" android:toXScale="0.4"
        android:fromYScale="1.5" android:toYScale="0.7"
        android:pivotX="50%" android:pivotY="65%" android:duration="4500"
        android:repeatCount="infinite" android:repeatMode="reverse" />
    <rotate
        android:fromDegrees="0.0"
        android:toDegrees="60.0"
        android:pivotX="50%"
        android:pivotY="50%" />
    <alpha
        android:fromAlpha="1.0" android:toAlpha="0.1" android:duration="2250"
        android:repeatCount="infinite" android:repeatMode="reverse" />
    <translate
        android:fromXDelta="0.0"
        android:toXDelta="50.0"
        android:fromYDelta="20.0"
        android:toYDelta="80.0"
        android:duration="2250"
        android:repeatMode="reverse"
        android:repeatCount="infinite" />
</set>
\end{lstlisting}

Здесь задействуются четыре типа анимаций: элемент scale представляет
масштабирование, элемент rotate --- вращение, элемент alpha --- изменение
прозрачности, элемент translate --- перемещение. Если бы мы использовали
одну анимацию, то могли бы определить один корневой элемент по типу
анимации. Но так как мы используем набор, то все анимации помещаются в
элемент set, который представляет класс AnimationSet.\par
Для всех анимаций также характерно указание начальной и конечной точки
трансформации.

\paragraph{Анимация масштабирования}
Для анимация масштабирования задается начальное масштабирование по оси
х (android:fromXScale) и по оси y (android:fromYScale) и конечные
значения масштабирования android:toXScale и android:toYScale. Например,
так как android:fromXScale=1.0, а android:toXScale=0.5, то по ширине будет
происходить сжатие на 50%.
Также при масштабировании устанавливаются зачения android:pivotX и
android:pivotY, которые указывают на центр масшабирования или опорную
точку.

\paragraph{Анимация вращения}
Для анимации вращения задается начальное (android:fromDegrees) и
конечное значения поворота (android:toDegrees).
С помощью свойств android:pivotX и android:pivotY также, как и при
масштабировании, задается опорная точка вращения.

\paragraph{Анимация прозрачности}
Для анимации прозрачности задается начальное значение прозрачности
(android:fromAlpha) и финальное значение, устанавливамое при завершении
анимации (android:toAlpha). Все значения варьируют в диапазоне от 1.0
(непрозрачный) до 0.0 (полностью прозрачный)

\paragraph{Анимация перемещения}
Для перемещения также устанавливаются начальные (android:fromXDelta и
android:fromYDelta) и конечные значения (android:toXDelta и
android:toYDelta)
Для всех анимаций начальные и конечные значения указывают некий
диапазон, в котором будут ранжироваться значения. Само вычисление
значений на этом промежутке зависит от конкретного алгоритма. В данном
случае в качестве алгоритма устанавливается линейная интерполяция. Для
этого у корневого элемента set определено свойство
android:interpolator="@android:anim/linear\_interpolator".

Данная анимация будет применяться к элементу ImageView, который
определим в файле разметки интерфейса:

\begin{lstlisting}[language=XML
	, label=lst:
	]
<?xml version="1.0" encoding="utf-8"?>
<androidx.constraintlayout.widget.ConstraintLayout xmlns:android="http://schemas.android.com/apk/res/android"
    xmlns:app="http://schemas.android.com/apk/res-auto"
    xmlns:tools="http://schemas.android.com/tools"
    android:layout_width="match_parent"
    android:layout_height="match_parent"
    tools:context=".TweenAnimationActivity">

    <ImageView android:id="@+id/animationView"
        android:layout_width="0dp"
        android:layout_height="0dp"
        android:adjustViewBounds="true"
        app:layout_constraintBottom_toBottomOf="parent"
        app:layout_constraintLeft_toLeftOf="parent"
        app:layout_constraintRight_toRightOf="parent"
        app:layout_constraintTop_toTopOf="parent"/>

</androidx.constraintlayout.widget.ConstraintLayout>
\end{lstlisting}

Теперь определим и запустим анимацию в классе:

\begin{lstlisting}[language=Java
	, label=lst:
	]
@Override
protected void onCreate(Bundle savedInstanceState) {
	super.onCreate(savedInstanceState);
	setContentView(R.layout.activity_tween_animation);

	ImageView img = findViewById(R.id.animationView);
	img.setImageResource(R.drawable.gal);
	Animation animation = AnimationUtils.loadAnimation(this,
			R.anim.common_animation);
	img.startAnimation(animation);
}
\end{lstlisting}

\section{Сервисы}
\subsection{Введение в сервисы Android}
Сервисы представляют собой особую организацию приложения. В отличие
от activity они не требуют наличия визуального интерфейса. Сервисы
позволяют выполнять долговременные задачи без вмешательства
пользователя.\par
Все сервисы наследуются от класса Service и проходят следующие этапы
жизненного цикла:
\begin{itemize}
	\item Метод onCreate(): вызывается при создании сервиса
	\item Метод onStartCommand(): вызывается при получении сервисом
		команды, отправленной с помощью метода startService()
	\item Метод onBind(): вызывается при закреплении клиента за сервисом с
		помощью метода bindService()
	\item Метод onDestroy(): вызывается при завершении работы сервиса
\end{itemize}

Для воспроизведения аудио-файла определим в классе MediaService
следующий код:

\begin{lstlisting}[language=Java
	, label=lst:
	]
@Override
protected void onCreate(Bundle savedInstanceState) {
	super.onCreate(savedInstanceState);
	setContentView(R.layout.activity_service);
}

public void click(View v) {
	Intent i = new Intent(this, MediaService.class);
	if (v.getId() == R.id.start) {
		startService(i);
	}
	else {
		stopService(i);
	}
}
\end{lstlisting}

Для запуска сервиса в классе Activity определен метод startService(), в
который передается объект Intent. Этот метод будет посылать команду
сервису и вызывать его метод onStartCommand(), а также указывать
системе, что сервис должен продолжать работать до тех пор, пока не будет
вызван метод stopService().\par
Метод stopService() также определен к классе Activity и принимает объект
Intent. Он останавливает работу сервиса, вызывая его метод onDestroy().\par
Чтобы управлять сервисом, изменим activity. Сначала добавим в файл
разметки пару кнопок для управления сервисом:

\begin{lstlisting}[language=XML
	, label=lst:
	]
<?xml version="1.0" encoding="utf-8"?>
<androidx.constraintlayout.widget.ConstraintLayout xmlns:android="http://schemas.android.com/apk/res/android"
    xmlns:app="http://schemas.android.com/apk/res-auto"
    xmlns:tools="http://schemas.android.com/tools"
    android:layout_width="match_parent"
    android:layout_height="match_parent"
    tools:context=".ServiceActivity">

    <Button
        android:id="@+id/start"
        android:layout_width="0dp"
        android:layout_height="wrap_content"
        android:text="Start"
        android:onClick="click"
        app:layout_constraintLeft_toLeftOf="parent"
        app:layout_constraintRight_toLeftOf="@id/stop"
        app:layout_constraintTop_toTopOf="parent" />
    <Button
        android:id="@+id/stop"
        android:layout_width="0dp"
        android:layout_height="wrap_content"
        android:text="Stop"
        android:onClick="click"
        app:layout_constraintLeft_toRightOf="@id/start"
        app:layout_constraintRight_toRightOf="parent"
        app:layout_constraintTop_toTopOf="parent" />

</androidx.constraintlayout.widget.ConstraintLayout>
\end{lstlisting}

И в конце нам надо зарегистрировать сервис в файле манифеста:

\begin{lstlisting}[language=XML
	, label=lst:
	]
<service
	android:name=".MediaService"
	android:enabled="true"
	android:exported="true" >
</service>
\end{lstlisting}

\section{Диалоговые окна}
\subsection{DatePickerDialog и TimePickerDialog}
По умолчанию в Android уже определены два диалоговых окна ---
DatePickerDialog и TimePickerDialog, которые позволяют выбрать дату и
время.\par
Кроме установки даты DatePickerDialog позволяет обработать выбор даты с
помощью слушателей OnDateChangedListener и OnDateSetListener. Что
позволяет использовать выбранную дату далее в приложении.\par
Подобным образом TimePickerDialog позволяет обработать выбор времени с
помощью слушателей OnTimeChangedListener и OnTimeSetListener.\par
Используем DatePickerDialog и TimePickerDialog в приложении. Определим
следующую разметку интерфейса в разметки:

\begin{lstlisting}[language=XML
	, label=lst:
	]
<?xml version="1.0" encoding="utf-8"?>
<androidx.constraintlayout.widget.ConstraintLayout xmlns:android="http://schemas.android.com/apk/res/android"
    xmlns:app="http://schemas.android.com/apk/res-auto"
    xmlns:tools="http://schemas.android.com/tools"
    android:layout_width="match_parent"
    android:layout_height="match_parent"
    tools:context=".DialogActivity">

    <TextView
        android:id="@+id/currentDateTime"
        android:layout_width="0dp"
        android:layout_height="wrap_content"
        android:textSize="20sp"
        app:layout_constraintBottom_toTopOf="@id/timeButton"
        app:layout_constraintLeft_toLeftOf="parent"
        app:layout_constraintRight_toRightOf="parent"
        app:layout_constraintTop_toTopOf="parent" />

    <Button
        android:id="@+id/timeButton"
        android:layout_width="0dp"
        android:layout_height="wrap_content"
        android:text="Изменить время"
        android:onClick="setTime"
        app:layout_constraintBottom_toTopOf="@id/dateButton"
        app:layout_constraintLeft_toLeftOf="parent"
        app:layout_constraintRight_toRightOf="parent"
        app:layout_constraintTop_toBottomOf="@id/currentDateTime"  />

    <Button
        android:id="@+id/dateButton"
        android:layout_width="0dp"
        android:layout_height="wrap_content"
        android:text="Изменить дату"
        android:onClick="setDate"
        app:layout_constraintLeft_toLeftOf="parent"
        app:layout_constraintRight_toRightOf="parent"
        app:layout_constraintTop_toBottomOf="@id/timeButton" />
</androidx.constraintlayout.widget.ConstraintLayout>
\end{lstlisting}

Здесь определены две кнопки для выбора даты и времени и текстовое поле,
отображающее выбранные дату и время. И изменим код:

\begin{lstlisting}[language=Java
	, label=lst:
	]
TextView currentDateTime;
Calendar dateAndTime = Calendar.getInstance();

@Override
public void onCreate(Bundle savedInstanceState) {
	super.onCreate(savedInstanceState);
	setContentView(R.layout.activity_dialog);
	currentDateTime = findViewById(R.id.currentDateTime);
	setInitialDateTime();
}

// отображаем диалоговое окно для выбора даты
public void setDate(View v) {
	new DatePickerDialog(DialogActivity.this, d,
			dateAndTime.get(Calendar.YEAR),
			dateAndTime.get(Calendar.MONTH),
			dateAndTime.get(Calendar.DAY_OF_MONTH))
			.show();
}

// отображаем диалоговое окно для выбора времени
public void setTime(View v) {
	new TimePickerDialog(DialogActivity.this, t,
			dateAndTime.get(Calendar.HOUR_OF_DAY),
			dateAndTime.get(Calendar.MINUTE), true)
			.show();
}
// установка начальных даты и времени
private void setInitialDateTime() {
	currentDateTime.setText(DateUtils.formatDateTime(this
			, dateAndTime.getTimeInMillis()
			,DateUtils.FORMAT_SHOW_DATE
					| DateUtils.FORMAT_SHOW_YEAR
					| DateUtils.FORMAT_SHOW_TIME
			));
}

// установка обработчика выбора времени
TimePickerDialog.OnTimeSetListener t = new TimePickerDialog.OnTimeSetListener() {
	public void onTimeSet(TimePicker view, int hourOfDay, int minute) {
		dateAndTime.set(Calendar.HOUR_OF_DAY, hourOfDay);
		dateAndTime.set(Calendar.MINUTE, minute);
		setInitialDateTime();
	}
};

// установка обработчика выбора даты
DatePickerDialog.OnDateSetListener d = new DatePickerDialog.OnDateSetListener() {
	public void onDateSet(DatePicker view, int year, int monthOfYear, int dayOfMonth) {
		dateAndTime.set(Calendar.YEAR, year);
		dateAndTime.set(Calendar.MONTH, monthOfYear);
		dateAndTime.set(Calendar.DAY_OF_MONTH, dayOfMonth);
		setInitialDateTime();
	}
};
\end{lstlisting}

Ключевым классом здесь является java.util.Calendar, который хранится в
стандартной библиоетке классов Java. В методе setInitialDateTime() мы
получаем из экземпляра этого класса количество миллисекунд
dateAndTime.getTimeInMillis() и с помощью форматирования выводим на
текстовое поле.\par
Метод setDate(), вызываемый по нажатию на кнопку, отображает окно для
выбора даты. При создании окна его объекту передается обработчик выбора
даты DatePickerDialog.OnDateSetListener, который изменяет дату на
текстовом поле.\par
Аналогично метод setTime() отображает окно для выбора времени. Объект
окна использует обработчик выбора времени
TimePickerDialog.OnTimeSetListener, который изменяет время на текстовом
поле.

\subsection{DialogFragment и создание своих диалоговых окон}
Для создания своих диалоговых окон используется компонент AlertDialog в
связке с классом фрагмента DialogFragment. Рассмотрим их применение.\par
Вначале добавим в проект новый класс фрагмента, который назовем
CustomDialogFragment:

\begin{lstlisting}[language=Java
	, label=lst:
	]
private Removable removable;

@Override
public void onAttach(Context context){
	super.onAttach(context);
	removable = (Removable) context;
}

@NonNull
public Dialog onCreateDialog(Bundle savedInstanceState) {
	String job = getArguments().getString("phone");
	AlertDialog.Builder builder=new AlertDialog.Builder(getActivity());
	return builder
			.setTitle("Диалоговое окно")
			.setIcon(android.R.drawable.ic_dialog_alert)
			.setMessage("Вы выбрали: " + job + "!!!")
			.setPositiveButton("OK", null)
			.setNegativeButton("Удалить"
				, ((dialog, which) -> removable.remove(job)))
			.create();
}
\end{lstlisting}

Класс фрагмента содержит всю стандартную функциональность фрагмента с
его жизненным циклом, но при этом наследуется от класса DialogFragment,
который добавляет ряд дополнительных функций. И для его создания мы
можем использовать два способа:

\begin{itemize}
	\item Переопределение метода \texttt{onCreateDialog()}, который возвращает
		объект Dialog.
	\item Использование стандартного метода\texttt{onCreateView()} 
\end{itemize}

Для вызова этого диалогового окна в файле разметки определим
кнопку:

\begin{lstlisting}[language=XML
	, label=lst:
	]
<?xml version="1.0" encoding="utf-8"?>
<androidx.constraintlayout.widget.ConstraintLayout xmlns:android="http://schemas.android.com/apk/res/android"
    xmlns:app="http://schemas.android.com/apk/res-auto"
    xmlns:tools="http://schemas.android.com/tools"
    android:layout_width="match_parent"
    android:layout_height="match_parent"
    tools:context=".DialogActivity">

    <TextView
        android:id="@+id/currentDateTime"
        android:layout_width="0dp"
        android:layout_height="wrap_content"
        android:textSize="20sp"
        app:layout_constraintBottom_toTopOf="@id/timeButton"
        app:layout_constraintLeft_toLeftOf="parent"
        app:layout_constraintRight_toRightOf="parent"
        app:layout_constraintTop_toTopOf="parent" />

    <Button
        android:id="@+id/timeButton"
        android:layout_width="0dp"
        android:layout_height="wrap_content"
        android:text="Изменить время"
        android:onClick="setTime"
        app:layout_constraintBottom_toTopOf="@id/dateButton"
        app:layout_constraintLeft_toLeftOf="parent"
        app:layout_constraintRight_toRightOf="parent"
        app:layout_constraintTop_toBottomOf="@id/currentDateTime"  />

    <Button
        android:id="@+id/dateButton"
        android:layout_width="0dp"
        android:layout_height="wrap_content"
        android:text="Изменить дату"
        android:onClick="setDate"
        app:layout_constraintLeft_toLeftOf="parent"
        app:layout_constraintRight_toRightOf="parent"
        app:layout_constraintTop_toBottomOf="@id/timeButton" />
</androidx.constraintlayout.widget.ConstraintLayout>
\end{lstlisting}

В коде определим обработчик нажатия кнопки, который будет
запускать диалоговое окно:

\begin{lstlisting}[language=Java
	, label=lst:
	]
TextView currentDateTime;
Calendar dateAndTime = Calendar.getInstance();

@Override
public void onCreate(Bundle savedInstanceState) {
	super.onCreate(savedInstanceState);
	setContentView(R.layout.activity_dialog);
	currentDateTime = findViewById(R.id.currentDateTime);
	setInitialDateTime();
}

// отображаем диалоговое окно для выбора даты
public void setDate(View v) {
	new DatePickerDialog(DialogActivity.this, d,
			dateAndTime.get(Calendar.YEAR),
			dateAndTime.get(Calendar.MONTH),
			dateAndTime.get(Calendar.DAY_OF_MONTH))
			.show();
}

// отображаем диалоговое окно для выбора времени
public void setTime(View v) {
	new TimePickerDialog(DialogActivity.this, t,
			dateAndTime.get(Calendar.HOUR_OF_DAY),
			dateAndTime.get(Calendar.MINUTE), true)
			.show();
}
// установка начальных даты и времени
private void setInitialDateTime() {
	currentDateTime.setText(DateUtils.formatDateTime(this
			, dateAndTime.getTimeInMillis()
			,DateUtils.FORMAT_SHOW_DATE
					| DateUtils.FORMAT_SHOW_YEAR
					| DateUtils.FORMAT_SHOW_TIME
			));
}

// установка обработчика выбора времени
TimePickerDialog.OnTimeSetListener t = new TimePickerDialog.OnTimeSetListener() {
	public void onTimeSet(TimePicker view, int hourOfDay, int minute) {
		dateAndTime.set(Calendar.HOUR_OF_DAY, hourOfDay);
		dateAndTime.set(Calendar.MINUTE, minute);
		setInitialDateTime();
	}
};

// установка обработчика выбора даты
DatePickerDialog.OnDateSetListener d = new DatePickerDialog.OnDateSetListener() {
	public void onDateSet(DatePicker view, int year, int monthOfYear, int dayOfMonth) {
		dateAndTime.set(Calendar.YEAR, year);
		dateAndTime.set(Calendar.MONTH, monthOfYear);
		dateAndTime.set(Calendar.DAY_OF_MONTH, dayOfMonth);
		setInitialDateTime();
	}
};
\end{lstlisting}

\subsection{Передача данных в диалоговое окно}
Передача данных в диаговое окно, как и в любой фрагмент, осуществляется с
помощью объекта Bundle.
Так, определим в файле разметки список ListView:

\begin{lstlisting}[language=XML
	, label=lst:
	]
<?xml version="1.0" encoding="utf-8"?>
<androidx.constraintlayout.widget.ConstraintLayout xmlns:android="http://schemas.android.com/apk/res/android"
    xmlns:app="http://schemas.android.com/apk/res-auto"
    xmlns:tools="http://schemas.android.com/tools"
    android:layout_width="match_parent"
    android:layout_height="match_parent"
    tools:context=".CustomDialogActivity">

    <ListView
        android:id="@+id/phonesList"
        android:layout_width="match_parent"
        android:layout_height="match_parent"

        app:layout_constraintRight_toRightOf="parent"
        app:layout_constraintTop_toTopOf="parent"
        app:layout_constraintLeft_toLeftOf="parent"
        app:layout_constraintBottom_toBottomOf="parent"
        />

</androidx.constraintlayout.widget.ConstraintLayout>
\end{lstlisting}

В классе определим для этого списка данные:

\begin{lstlisting}[language=Java
	, label=lst:
	]
private ArrayAdapter<String> adapter;

@Override
protected void onCreate(Bundle savedInstanceState) {
	super.onCreate(savedInstanceState);
	setContentView(R.layout.activity_custom_dialog);

	ListView phonesList = findViewById(R.id.phonesList);
	ArrayList<String> jobs = new ArrayList<>();
	jobs.add("Бортмеханик");
	jobs.add("Гитарист");
	jobs.add("Вирусолог");
	jobs.add("Физиотерапевт");
	jobs.add("Художник по свету");
	jobs.add("Востоковед");
	jobs.add("Парикмахер");

	adapter = new ArrayAdapter<>(this, android.R.layout.simple_list_item_1, jobs);
	phonesList.setAdapter(adapter);

	phonesList.setOnItemClickListener(new AdapterView.OnItemClickListener() {
		@Override
		public void onItemClick(AdapterView<?> parent
			, View view, int position, long id) {
			String selectedJob = adapter.getItem(position);
			CustomDialogFragment dialog = new CustomDialogFragment();
			Bundle args = new Bundle();
			args.putString("phone", selectedJob);
			dialog.setArguments(args);
			dialog.show(getSupportFragmentManager(), "custom");
		}
	});
}

@Override
public void remove(String name) {
	adapter.remove(name);
}
\end{lstlisting}

С помощью метода getArguments() получаем переданный в
объект Bundle. И так как была передана строка, то для ее извлечения
применяется метод getString().

\subsection{Взаимодействие диалогового окна с Activity}
Взаимодействие между Activity и фрагментом производится, как правило,
через интерфейс. К примеру, в прошлой теме выводила список
объектов, и теперь определим удаление из этого списка через диалоговое
окно.
Для этого добавим в проект интерфейс Removable:

\begin{lstlisting}[language=Java
	, label=lst:
	]
public interface Removable {
	void remove(String name);
}
\end{lstlisting}

Единственный метод интерфейса remove получает удаляемый объект в виде
параметра name.
Теперь реализуем этот интерфейс в коде:

\begin{lstlisting}[language=Java
	, label=lst:
	]
private ArrayAdapter<String> adapter;

@Override
protected void onCreate(Bundle savedInstanceState) {
	super.onCreate(savedInstanceState);
	setContentView(R.layout.activity_custom_dialog);

	ListView phonesList = findViewById(R.id.phonesList);
	ArrayList<String> jobs = new ArrayList<>();
	jobs.add("Бортмеханик");
	jobs.add("Гитарист");
	jobs.add("Вирусолог");
	jobs.add("Физиотерапевт");
	jobs.add("Художник по свету");
	jobs.add("Востоковед");
	jobs.add("Парикмахер");

	adapter = new ArrayAdapter<>(this, android.R.layout.simple_list_item_1, jobs);
	phonesList.setAdapter(adapter);

	phonesList.setOnItemClickListener(new AdapterView.OnItemClickListener() {
		@Override
		public void onItemClick(AdapterView<?> parent, View view, int position, long id) {

			String selectedJob = adapter.getItem(position);
			CustomDialogFragment dialog = new CustomDialogFragment();
			Bundle args = new Bundle();
			args.putString("phone", selectedJob);
			dialog.setArguments(args);
			dialog.show(getSupportFragmentManager(), "custom");
		}
	});
}

@Override
public void remove(String name) {
	adapter.remove(name);
}
\end{lstlisting}

\clearpage

\section*{\LARGE{Вывод}}
\addcontentsline{toc}{section}{Вывод}
В данной практической работе были получены знания о работе с видео и аудио.
В том чилсе загружать контент из интернета.
Также научились создавать анимация такую как: cell-анимация и tween-анимация.
Научились создавать сервиси и различные диалоговые окна.


