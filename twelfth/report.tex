\graphicspath{{./twelfth/img}}

\section*{\LARGE{Введение}}
\addcontentsline{toc}{section}{Введение}

\textbf{Задание}:
\begin{itemize}
	\item
1. Реализовать приложение для работы с видео материалами в
стандартном наборе виджетов применяя класс VideoView
2. С помощью класса MediaController добавить к VideoView
дополнительно элементы управления.
3. Реализовать воспроизведение файла из интернета.
4. Реализовать приложение для работы с аудио материалами, применяя
класс MediaPlayer
5. Реализовать приложение для работы с анимацией, применяя
технологию Cell animation.
6. Реализовать приложение для работы с анимацией, применяя
технологию Tween- animation. Создать приложение с сервисом со
всеми необходимыми этапами жизненного цикла, применяя класс
Service.
7. Реализовать диалоговые окна, применяя технологии DatePickerDialog и
TimePickerDialog, которые позволяют выбрать дату и время.
8. Применяя класс DialogFragment создать свои диалоговые окна.
9. Реализовать передачу данных в диалоговое окно, как и в любой
фрагмент, с помощью объекта Bundle.
10.Реализовать взаимодействие диалогового окна с Activity
\end{itemize}

\clearpage

\section*{\LARGE{Выполнение практической работы}}
\addcontentsline{toc}{section}{Выполнение практической работы}

\section{}

\begin{lstlisting}[language=Java
	, label=lst:
	]
\end{lstlisting}

\clearpage

\section*{\LARGE{Вывод}}
\addcontentsline{toc}{section}{Вывод}
В данной практической работе были получены знания об создании и управлении
потоками. Научились работать как с объектами Thread, так и с AsyncTask.
Узнали как работать с ViewModel и как применять их со фрагментами.\par
Также узнали как работать с сетью. Познакомились с виджетом WebView,
и научились с помощью него открывать страницы из интернета.

