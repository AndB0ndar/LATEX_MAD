\graphicspath{{./eighth/img}}

\section*{\LARGE{Введение}}
\addcontentsline{toc}{section}{Введение}
Организация приложения на основе нескольких activity не всегда может быть
оптимальной. Мир ОС Android довольно сильно фрагментирован и состоит
из многих устройств. И если для мобильных аппаратов с небольшими
экранами взаимодействие между разными activity выглядит довольно
неплохо, то на больших экранах --- планшетах, телевизорах окна activity
смотрелись бы не очень в силу большого размера экрана. Собственно
поэтому и появилась концепция фрагментов.\par
Фрагмент представляет кусочек визуального интерфейса приложения,
который может использоваться повторно и многократно. У фрагмента может
быть собственный файл layout, у фрагментов есть свой собственный
жизненный цикл. Фрагмент существует в контексте activity и имеет свой
жизненный цикл, вне activity обособлено он существовать не может. Каждая
activity может иметь несколько фрагментов.\par
Для работы с фрагментами необходимо добавить в файл build.gradle строку:

\begin{verbatim}
implementation "androidx.fragment:fragment:1.3.6"
\end{verbatim}

\clearpage

\section*{\LARGE{Выполнение практической работы}}
\addcontentsline{toc}{section}{Выполнение практической работы}

\section{Добавление фрагмента в Activity}
Для использования фрагмента добавим его в MainActivity. Для этого
изменим файл activity\_main.xml, которая определяет интерфейс для
MainActivity (Листнинг~\ref{lst:xml:add:fragment}).
\begin{lstlisting}[language=XML
	, caption=\leftline{}
	, label=lst:xml:add:fragment
	]
<androidx.fragment.app.FragmentContainerView
	xmlns:android="http://schemas.android.com/apk/res/android"
	android:id="@+id/fragment_container_view"
	android:layout_width="match_parent"
	android:layout_height="match_parent"
	android:name="com.example.fragmentapp.ContentFragment" />
\end{lstlisting}

Для добавления фрамента применяется элемент FragmentContainerView. По
сути, FragmentContainerView представляет объект View, который расширяет
класс FrameLayout и предназначен специально для работы с фрагментами.
Собственно кроме фрагментов он больше ничего содержать не может.\par
Его атрибут \texttt{android:name} указывает на имя класса фрагмента, который
будет использоваться. В моем случае полное имя класса фрагмента с учетов
пакета \texttt{com.example.fragmentapp.ContentFragment}.\par
Код класса фрагмента показан на листнинг~\ref{lst:java:fragment:class}

\begin{lstlisting}[language=Java
	, caption=\leftline{}
	, label=lst:java:fragment:class
	]
public class ContentFragment extends Fragment {
	public ContentFragment(){
	super(R.layout.fragment_content);
	}
}
\end{lstlisting}

Код класса MainActivity остается тем же, что и при создании проекта.

\section{Добавление логики к фрагменту}
Пример использования логики в фрагментах показан
в листнинге~\ref{lst:xml:fragment:action}.

\begin{lstlisting}[language=XML
	, caption=\leftline{}
	, label=lst:xml:fragment:action
	]
private static int count = 0;

@Override
public void onViewCreated(@NonNull View view, @Nullable Bundle
		savedInstanceState) {
	super.onViewCreated(view, savedInstanceState);
	Button updateButton = view.findViewById(R.id.updateButton);
	TextView updateBox = view.findViewById(R.id.dateTextView);
	updateBox.setText(String.valueOf(count));
	updateButton.setOnClickListener(new View.OnClickListener() {
		@Override
		public void onClick(View v) {
			++count;
			updateBox.setText(String.valueOf(count));
		}
	});
}
\end{lstlisting}

\section{Добавление фрагмента в коде}
Кроме определения фрагмента в xaml-файле интерфейса мы можем добавить
его динамически в activity.\par
Для этого изменим файл activity\_main.xml как показано в
листнинге~\ref{lst:xml:frg:java}.

\begin{lstlisting}[language=XML
	, caption=\leftline{}
	, label=lst:xml:frg:java
	]
<androidx.fragment.app.FragmentContainerView
	xmlns:android="http://schemas.android.com/apk/res/android"
	android:id="@+id/fragment_container_view"
	android:layout_width="match_parent"
	android:layout_height="match_parent" />
\end{lstlisting}

И также изменим класс MainActivit как показано
в листнинге~\ref{lst:java:frg:java}.

\begin{lstlisting}[language=Java
	, caption=\leftline{Реализация фаргмента в Java коде}
	, label=lst:java:frg:java
	]
@Override
protected void onCreate(Bundle savedInstanceState) {
	super.onCreate(savedInstanceState);
	setContentView(R.layout.activity_main);
	if (savedInstanceState == null) {
		getSupportFragmentManager().beginTransaction()
				.add(R.id.fragment_container_view
					, ContentFragment.class
					, null)
				.commit();
	}
}
\end{lstlisting}

Метод \texttt{getSupportFragmentManager()} возвращает объект FragmentManager,
который управляет фрагментами.\par
Объект FragmentManager с помощью метода \texttt{beginTransaction()} создает
объект FragmentTransaction.\par
FragmentTransaction выполняет два метода: \texttt{add()} и \texttt{commit()}.
Метод \texttt{add()} добавляет фрагмент:
\texttt{add(R.id.fragment\_container\_view, new 
ContentFragment())} --- первым аргументом передается ресурс разметки, в
который надо добавить фрагмент (это определенный в activity\_main.xml
элемент \texttt{androidx.fragment.app.FragmentContainerView}).
И метод \texttt{commit()} подтвержает и завершает операцию добавления.\par
Итоговый результат такого добавления фрагмента будет тем же, что и при
явном определении фрагмента через элемент \texttt{FragmentContainerView} в
разметке интерфейса.

\section{Взаимодействие между фрагментами}
Одна activity может использовать несколько фрагментов, например, с одной
стороны список, а с другой --- детальное описание выбранного элемента
списка. В такой конфигурации activity использует два фрагмента, которые
между собой должны взаимодействовать. Рассмотрим базовые принципы
взаимодействия фрагментов в приложении.\par
Разметка первого фрагмента показана в листнинге~\ref{lst:xml:frg:first}.

\begin{lstlisting}[language=XML
	, caption=\leftline{}
	, label=lst:xml:frg:first
	]
<?xml version="1.0" encoding="utf-8"?>
<androidx.constraintlayout.widget.ConstraintLayout
    xmlns:android="http://schemas.android.com/apk/res/android"
    xmlns:app="http://schemas.android.com/apk/res-auto"
    android:layout_width="match_parent"
    android:layout_height="match_parent"
    android:padding="16dp">
    <ListView
        android:id="@+id/countriesList"
        android:layout_width="0dp"
        android:layout_height="0dp"
        app:layout_constraintBottom_toBottomOf="parent"
        app:layout_constraintLeft_toLeftOf="parent"
        app:layout_constraintRight_toRightOf="parent"
        app:layout_constraintTop_toTopOf="parent">
    </ListView>
</androidx.constraintlayout.widget.ConstraintLayout>
\end{lstlisting}

Разметка второго фрагмента показана в листнинге~\ref{lst:xml:frg:second}.

\begin{lstlisting}[language=XML
	, caption=\leftline{}
	, label=lst:xml:frg:second
	]
<?xml version="1.0" encoding="utf-8"?>
<androidx.constraintlayout.widget.ConstraintLayout
    xmlns:android="http://schemas.android.com/apk/res/android"
    xmlns:app="http://schemas.android.com/apk/res-auto"
    android:layout_width="match_parent"
    android:layout_height="match_parent"
    android:padding="16dp">
    <TextView
        android:id="@+id/detailsText"
        android:layout_width="wrap_content"
        android:layout_height="wrap_content"
        android:layout_gravity="center"
        android:text="No select"
        android:textSize="22sp"
        app:layout_constraintLeft_toLeftOf="parent"
        app:layout_constraintTop_toTopOf="parent"
        />
</androidx.constraintlayout.widget.ConstraintLayout>
\end{lstlisting}

Фрагменты не могут напрямую взаимодействовать между собой. Для этого
надо обращаться к контексту, в качестве которого выступает класс Activity.
Для обращения к activity, как правило, создается вложенный интерфейс. В
данном случае он называется \texttt{OnFragmentSendDataListener} с одним
методом:

\begin{verbatim}
interface OnFragmentSendDataListener {
	void onSendData(String data);
}
private OnFragmentSendDataListener fragmentSendDataListener;
\end{verbatim}

Но чтобы взаимодействовать с другим фрагментом через activity, нам надо
прикрепить текущий фрагмент к activity. Для этого в классе фрагмента
определен метод onAttach(Context context). В нем происходит установка
объекта OnFragmentSendDataListener:

\begin{verbatim}
fragmentSendDataListener = (OnFragmentSendDataListener) context;
\end{verbatim}

При обработке нажатия на элемент в списке мы можем отправить Activity
данные о выбранном объекте:

\begin{verbatim}
String selectedItem = (String)parent.getItemAtPosition(position);
fragmentSendDataListener.onSendData(selectedItem);
\end{verbatim}

Таким образом, при выборе объекта в списке MainActivity получит
выбранный объект.

И в конце изменим код MainActivity, как показано
в листнинге~\ref{lst:interaction:main}.

\begin{lstlisting}[language=Java
	, caption=\leftline{}
	, label=lst:interaction:main
	]
public class MainActivity
	extends AppCompatActivity
	implements ListFragment.OnFragmentSendDataListener {
	@Override
	protected void onCreate(Bundle savedInstanceState) {
		super.onCreate(savedInstanceState);
		setContentView(R.layout.activity_main_1_7);
	}

	@Override
	public void onSendData(String selectedItem) {
		DetailFragment fragment = (DetailFragment) getSupportFragmentManager()
				.findFragmentById(R.id.detailFragment);
		if (fragment != null)
			fragment.setSelectedItem(selectedItem);
	}
}
\end{lstlisting}

Для взаимодействия фрагмента ListFragment c другим фрагментом через
MainActivity надо, чтобы эта activity реализовывала интерфейс
\texttt{OnFragmentSendDataListener}. Для этого реализуем метод
\texttt{onSendData()},
который получает фрагмент \texttt{DetailFragment} и вызывает у него метод
\texttt{setSelectedItem()}.\par
В итоге получится, что при выборе в списке во фрагменте ListFragment будет
срабатывать слушатель списка и в частности его метод
\texttt{onItemClick(AdapterView<?> parent, View v, int position, long id)},
который вызовет метод
\texttt{fragmentSendDataListener.onSendData(selectedItem);}.
fragmentSendDataListener устанавливается как MainActivity, поэтому при
этом будет вызван метод setSelectedItem у фрагмента DetailFragment. Таким
образом, произойдет взаимодействие между двумя фрагментами.

\section{Фрагменты в альбомном и портретном режиме}
В прошлом материале было разработано приложение, которое выводит оба
фрагмента на экран. Продолжим работу с этим проектом. Всего было создано
два фрагмента: ListFragment для отображения списка и DetailFragment для
отображения выбранного элемента в списке. И MainActivity выводила оба
фрагмента на экран.\par
Но отображение двух и более фрагментов при портретной ориентации не
очень оптимально. Например, в прошлом материале приложение выглядело
так.\par
Но если список большой, то второй фрагмент, который отображает
выбранный элемент, соответственно уходит вниз. При альбомной
ориентации получится расположение еще более неоптимально. Поэтому
сначала изменим файл activity\_main.xml, чтобы более удобно располагать
фрагменты в альбомной ориентации, как показано
в листнинге~\ref{lst:xml:frg:orient}.

\begin{lstlisting}[language=XML
	, caption=\leftline{}
	, label=lst:xml:frg:orient
	]
<?xml version="1.0" encoding="utf-8"?>
<androidx.constraintlayout.widget.ConstraintLayout
    xmlns:android="http://schemas.android.com/apk/res/android"
    xmlns:app="http://schemas.android.com/apk/res-auto"
    android:layout_width="match_parent"
    android:layout_height="match_parent" >
    <androidx.fragment.app.FragmentContainerView
        android:id="@+id/listFragment"
		android:layout_width="0dp"
		android:layout_height="0dp"
		android:name="com.example.eighth.ListFragment"
		app:layout_constraintLeft_toLeftOf="parent"
		app:layout_constraintTop_toTopOf="parent"
		app:layout_constraintRight_toLeftOf="@id/detailFragment"
		app:layout_constraintBottom_toBottomOf="parent" />
    <androidx.fragment.app.FragmentContainerView
        android:id="@+id/detailFragment"
        android:layout_width="0dp"
        android:layout_height="0dp"
        android:name="com.example.eighth.DetailFragment"
        app:layout_constraintLeft_toRightOf="@id/listFragment"
        app:layout_constraintTop_toTopOf="parent"
        app:layout_constraintBottom_toBottomOf="parent"
        app:layout_constraintRight_toRightOf="parent"/>
</androidx.constraintlayout.widget.ConstraintLayout>
\end{lstlisting}

Теперь рассмотрим, как более удачно расположить фрагменты при
портретной ориентации. Нередко в этом случае решение следующее ---
одномоментно экран отображает только один фрагмент.\par
Итак, создадим в проекте в каталоге \textbf{res}, где хранятся все ресурсы,
подкаталог \textbf{layout-port}, который будет хранить файлы интерфейса
для портретной ориентации. Для этого переключимся к полному виду проекта.
Нажмем на папку res правой кнопкой мыши и в контекстном меню выберем
\textbf{New -> Android Resource Directory}.
Далее добавим в res/layout-port новый файл activity\_main.xml и определим в
нем код, как показано в листнинге~\ref{lst:xml:actwthfrg}.

\begin{lstlisting}[language=XML
	, caption=\leftline{}
	, label=lst:xml:actwthfrg
	]
<androidx.fragment.app.FragmentContainerView
	xmlns:android="http://schemas.android.com/apk/res/android"
	android:id="@+id/listFragment"
	android:layout_width="match_parent"
	android:layout_height="match_parent"
	android:name="com.example.fragmentapp.ListFragment" />
\end{lstlisting}

Этот файл будет использоваться для портретной ориентации MainActivity.
Таким образом, MainActivity в портретном режиме будет отображать только
один список.\par
Теперь добавим новую activity, которую назовем DetailActivity.\par
Таким образом, интерфейс DetailActivity будет определяться загружаемым
фрагментом DetailFragment.\par
Далее в коде DetailActivity определим код, как показано
в листнинге~\ref{lst:java:frg:orient}.

\begin{lstlisting}[language=Java
	, caption=\leftline{}
	, label=lst:java:frg:orient
	]
@Override
protected void onCreate(Bundle savedInstanceState) {
	super.onCreate(savedInstanceState);
	if (getResources().getConfiguration().orientation ==
			Configuration.ORIENTATION_LANDSCAPE) {
		finish();
		return;
	}
	setContentView(R.layout.activity_detail);
	Bundle extras = getIntent().getExtras();
	if (extras != null)
		selectedItem = extras.getString(SELECTED_ITEM);
}

@Override
protected void onResume() {
	super.onResume();
	DetailFragment fragment = (DetailFragment) getSupportFragmentManager()
			.findFragmentById(R.id.detailFragment);
	if (fragment != null)
		fragment.setSelectedItem(selectedItem);
}
\end{lstlisting}

Код MainActivity при этом будет выглядеть как показано
в листнинг~\ref{lst:java:main:orient}.

\begin{lstlisting}[language=Java
	, caption=\leftline{}
	, label=lst:java:main:orient
	]
@Override
public void onSendData(String selectedItem) {
	DetailFragment fragment = (DetailFragment) getSupportFragmentManager()
			.findFragmentById(R.id.detailFragment);
	if (fragment != null && fragment.isVisible())
		fragment.setSelectedItem(selectedItem);
	else {
		Intent intent = new Intent(getApplicationContext()
				, DetailActivity.class);
		intent.putExtra(DetailActivity.SELECTED_ITEM, selectedItem);
		startActivity(intent);
	}
}
\end{lstlisting}

\clearpage

\section*{\LARGE{Вывод}}
\addcontentsline{toc}{section}{Вывод}
В этой практической работе были получены знания об создании и
упарвлении фрагментами. А также научились организовавыть взаимодействие
между ними.\par
Также с помощью фрагментов создали приложение, которое можно использовать
в вертикальном и горизонтальном положениях телефона.

